\chapter{Java graph ADT}
\label{app:javagraph}

This appendix presents a
simplified abstract class for representing a graph abstract data type
(ADT).  Although it is fully functional, it purposely omits most exception
handling and other niceties that should be in any commercial level
package.  These details would distract from our overall (introductory)
goal of showing how to implement a basic graph class in Java.\\

Our plan is to have a common data structure that represents both graphs
and digraphs.  A graph will be a digraph with anti-parallel arcs; that
is, if $(u,v) \in E$ then $(v,u) \in E$ also. The initial abstract class
presented below requires a core set of methods needed for the realized
graph ADT.  It will be extended with the actual internal data structure
representation in the form of adjacency matrix or adjacency lists (or
whatever the designer picks).

{\renewcommand{\ttdefault}{pcr} % for verbatim environment
\footnotesize%
\begin{verbatim}
package graphADT;

import java.util.ArrayList;
import java.io.BufferedReader;

/*
 * Current Abstract Data Type interface for (di)graph classes.
 */
public interface Graph 
{
 // Need default, copy and BufferedReader constructors
 // (commented since Java doesn't allow abstract constructors!)
 //
 // public GraphADT(); 
 // public GraphADT(GraphADT); 
 // public GraphADT(BufferedReader in);
\end{verbatim}%
}

Right from the beginning we get in trouble since Java does not allow
abstract constructors.   We will leave these as comments and hope the
graph class designer will abide by them.   We want to create graphs
from an empty graph, copy an existing graph, or read in one from some
external source.   In the case of a  \verb|BufferedReader| constructor
the user has to attach one to a string,  file or keyboard.   We will
see examples later.\\

We now proceed by presenting the alteration methods required for our 
graph class interface.

{\renewcommand{\ttdefault}{pcr} % for verbatim environment
\footnotesize \begin{verbatim}
  // data structure modifiers
  //
  void addVertices(int i);       // Add some vertices
  void removeVertex(int i);      // Remove vertex

  void addArc(int i, int j);     // Add directed edge
  void removeArc(int i, int j);  // Remove directed edge

  void addEdge(int i, int j);    // Add undirected edge
  void removeEdge(int i, int j); // Remove undirected edge
\end{verbatim}%
}

This small set of methods allows one to build the graph.  We will soon
explicitly define the methods for adding or deleting edges in terms of
the two arc methods.  An extended class can override these to improve
efficiency if it wants.   We now list a few methods for extracting
information from a graph object.

{\renewcommand{\ttdefault}{pcr} % for verbatim environment
\footnotesize \begin{verbatim}
// data structure queries
//
boolean isArc(int i, int j);    // Check for arcs
boolean isEdge(int i, int j);   // Check for edges

int degree(int i);        // Number of neighbours (outgoing)
int inDegree(int i);      // Number of incoming arcs

ArrayList<Integer> neighbours(int i);  // List of (out-)neighbours

int order();                 // Number of vertices
int size();                  // Number of edges

// output (default same as representation)
//
String toString();

} // end of interface Graph
\end{verbatim}%
}

For our implementation, we want a vertex's degree to equal the number of
vertices returned by the \verb|neighbours| method, which in our
implementations will correspond to \verb|degree(i)|.
Also, the method \verb|isEdge(i,j)| will most likely just check
whether \verb|isArc(i,j) && isArc(j,i)| is true.

Finally, one nice thing to offer is a method to view/save/print a graph. 
Traditionally in Java we define a \verb|toString| method for this.
Our two actual implementations will return human viewable adjacency
lists or adjacency matrices, depending on the internal representation.\\

%Notice how we went ahead and defined both the adjacency matrix and
%adjacency lists output methods.  
We have the \verb|toString| method
as an interface requirement for the derived classes to define.   We want a
\verb|BufferedReader| constructor for a graph class to accept its own
\verb|toString| output. Two common external graph representations
are handled by the methods given below.

\label{graphoutput}
{\renewcommand{\ttdefault}{pcr} % for verbatim environment
\footnotesize \begin{verbatim}
public String toStringAdjMatrix()
{
 StringBuffer o = new StringBuffer();
 o.append(order()+"\n");

 for( int i = 0; i < order(); i++ )
 {
    for( int j = 0; j < order(); j++ )
    {
       if ( isArc(i,j) ) o.append("1 ");
       else o.append("0 ");
    }
    o.append("\n");
 }
 return o.toString();
}

public String toStringAdjLists()
{
 StringBuffer o = new StringBuffer();
 o.append(order()+"\n");

 for( int i = 0; i < order(); i++ )
 {
    for( int j = 0; j < order(); j++ )
    {
       if ( isArc(i,j) ) o.append(j+" ");
    }
    o.append("\n");
 }
 return o.toString();
}
\end{verbatim}%
}

To make things convenient for ourselves we require that the first line of
our (two) external graph representations contain the number of vertices.
Strictly speaking, this is not needed for an 0/1 adjacency matrix.
This makes our parsing job easier and this  format allows us to store
more than one graph per input stream. (We can terminate a stream of
graphs with a sentinel graph of order zero.)

\section{Java adjacency matrix implementation}

We now define our first usable graph class based on an adjacency matrix
representation (for graphs and digraphs). This class extends our graph
interface \verb|Graph|.

{\renewcommand{\ttdefault}{pcr} % for verbatim environment
\footnotesize \begin{verbatim}
package graphADT;

import java.io.*;
import java.util.*;

/*  Current implementation uses adjacency matrix form of a graph.
 */
public class GraphAdjMatrix implements Graph
{

    //  Internal Representation and Constructors
    //
    protected int order;            // Number of vertices
    protected boolean adj[][];      // Adjacency matrix of graph

    public GraphAdjMatrix()         // default constructor
    {
        order = 0;
    }

    public GraphAdjMatrix(GraphAdjMatrix G) // copy constructor
    {
        int n = G.order();
        if ( n>0 ) { adj = new boolean[n][n]; order = n; }

        for (int i = 0; i < n; i++)
            for (int j = 0; j < n; j++)
                adj[i][j] = G.adj[i][j];
    }
    public GraphAdjMatrix(Graph G) // conversion constructor
    {
        int n = G.order();
        if ( n>0 ) { adj = new boolean[n][n]; order = n; }

        for (int i = 0; i < n; i++)
            for (int j = 0; j < n; j++)
                if (G.isArc(i,j)) adj[i][j] = true;
    }
\end{verbatim}%
}


%We isolated a private method \verb|_allocate| that gets memory for 
%the adjacency matrix.  
The default constructor simply creates an empty graph and thus there is 
no need to allocate any space.  The two copy constructors simply copy 
onto a new $n$-by-$n$ matrix the boolean adjacency values of the old graph.  
Notice that we want new storage and not an object reference for the copy.\\

%\begin{note}
%We use the programming convention of
%beginning private variables and methods with the underscore character.
%\end{note}
%
An alternative implementation (as given in the first edition of this
textbook) would also keep an integer variable \verb|space| to represent 
the total space allocated.  Whenever we delete vertices we do not want 
to reallocate a new matrix but to reshuffle the entries into the upper 
sub-matrix.
Then whenever adding more  vertices we just extend the dimension of
the sub-matrix.

Our last input constructor for \verb|GraphAdjMatrix| is now given.

{\renewcommand{\ttdefault}{pcr} % for verbatim environment
\footnotesize \begin{verbatim}
public GraphAdjMatrix(BufferedReader buffer)
{
  try
  {
      String line = buffer.readLine().trim();
      String[] tokens = line.split("\\s+");
  
      if (tokens.length != 1)
      {
        throw new Error("bad format: number of vertices");
      }
      int n = order = Integer.parseInt(tokens[0]);
  
      if ( n>0 ) adj = new boolean[n][n];
  
      for (int i = 0; i < n; i++)
      {
         line = buffer.readLine().trim();
         tokens = line.split("\\s+");
         if (tokens.length != n)
         {
            throw new Error("bad format: adjacency matrix");
         }
         
         for (int j = 0; j < n; j++)
         {
            int entry = Integer.parseInt(tokens[j]);
            adj[i][j] = entry != 0;
         }
       }
  }
   catch (IOException x) 
   { throw new Error("bad input stream"); }
}

\end{verbatim}%
}

We have tried to minimize the complexity of this \verb|BufferedReader| constructor.   We do however
throw a couple of errors if something does go wrong.   Otherwise, this method simply reads in an integer
$n$ denoting the dimension of the adjacency matrix and then reads in the 0/1 matrix.  Notice how the use
of the \verb|String.split| method to extract the integer inputs.\\


We next define several methods for altering this graph data structure.
The first two methods allow the user to add or delete vertices from a
graph.


{\renewcommand{\ttdefault}{pcr} % for verbatim environment
\footnotesize \begin{verbatim}
    //  Mutator Methods
    //
    public void addVertices(int n)
    {
        assert(0 <= n );
        boolean matrix[][] = new boolean[order+n][order+n];

        for (int i = 0; i < order; i++)
        {
            for (int j = 0; j < order; j++)
            {
                matrix[i][j] = adj[i][j];
            }
        }
        order += n;
        adj = matrix;
    }

    public void removeVertex(int v)
    {
        assert(0 <= v && v < order);
        order--;

        for (int i = 0; i < v; i++)
        {
            for (int j = v; j < order; j++)
            {
                adj[i][j] = adj[i][j+1];
            }
        }

        for (int i = v; i < order; i++)
        {
            for (int j = 0; j < v; j++)
            {
                adj[i][j] = adj[i+1][j];
            }
            for (int j = v; j < order; j++)
            {
                adj[i][j] = adj[i+1][j+1];
            }
        }
    }
\end{verbatim}%
}

\if 01
These two methods allow the user to add or delete vertices from a graph.  If we have already allocated
enough space the \verb|addVertices| method will simply expand the current size (while initializing
entries to \verb|false|).   Otherwise, a new larger matrix is allocated  and a copy is done.\\
\fi



The \verb|removeVertex| method is somewhat complicated in that we have to remove a row and column from
the matrix corresponding to the vertex being deleted.  We decided to do this in two passes. The first
pass (for variable $i < v$) simply shifts all column indices $j\geq v$ to the left.  The second pass
(for variable $i \ge v$) has to shift entries up by one while also shifting column indices $j
\geq v$ to the left.  The user of the graph should realize that the
indices of the vertices change!\\

Next, we have four relatively trivial methods for adding and deleting
arcs (and edges). Like the mutator methods for checking for valid vertex
indices we add some important \verb|assert| statements that can be
turned on with an option to the java compiler for debugging graph
algorithms.

{\renewcommand{\ttdefault}{pcr} % for verbatim environment
\footnotesize \begin{verbatim}
   // Mutator Methods (cont.)

    public void addArc(int i, int j)
    {
        assert(0 <= i && i < order);
        assert(0 <= j && j < order);
        adj[i][j] = true;
    }

    public void removeArc(int i, int j)
    {
        assert(0 <= i && i < order);
        assert(0 <= j && j < order);
        adj[i][j] = false;
    }

    public void addEdge(int i, int j)
    {
        assert(0 <= i && i < order);
        assert(0 <= j && j < order);
        adj[i][j] = adj[j][i] = true;
    }

    public void removeEdge(int i, int j)
    {
        assert(0 <= i && i < order());
        assert(0 <= j && j < order());
        adj[i][j] = adj[j][i] = false;
    }
\end{verbatim}%
}


The methods to access properties of the graph are also pretty straightforward.

{\renewcommand{\ttdefault}{pcr} % for verbatim environment
\footnotesize \begin{verbatim}
    //  Access Methods
    //
    public boolean isArc(int i, int j)
    {
        assert(0 <= i && i < order);
        assert(0 <= j && j < order);
        return adj[i][j];
    }

    public boolean isEdge(int i, int j)
    {
        return isArc(i,j) && isArc(j,i);
    }

    public int degree(int i) // row count
    {
        assert(0 <= i && i < order);
        int sz = 0;
        for (int j = 0; j < order; j++)
        {
            if (adj[i][j]) sz++;
        }
        return sz;
    }

    public int inDegree(int i) // column count
    {
        assert(0 <= i && i < order);
        int sz = 0;
        for (int j = 0; j < order; j++)
        {
            if (adj[j][i]) sz++;
        }
        return sz;
    }
\end{verbatim}%
}

Our constant-time method for checking whether an arc is present in a graph is given above in the method
\verb|isArc|.   Unfortunately, we have to check all neighbours for computing the in- and out- degrees.
Also the method, given below, for returning a list of neighbours for a vertex will need to scan all
potential vertices.

{\renewcommand{\ttdefault}{pcr} % for verbatim environment
\footnotesize \begin{verbatim}
    public ArrayList<Integer> neighbours(int i)
    {
        assert(0 <= i && i < order);
        ArrayList<Integer> nbrs = new ArrayList<Integer>();

        for (int j = 0; j < order; j++)
        {
            if (adj[i][j]) nbrs.add(j);
        }

        return nbrs;
    }


    public int order()
    {
        return order;
    }

    public int size() // Number of arcs (edges count twice)
    {
        int sz = 0;
//      boolean undirected = true;
        for (int i = 0; i< order; i++)
        {
            for (int j = 0; j< order; j++)
            {
                if ( adj[i][j]) sz++;
//              if ( adj[i][j] != adj[j][i] ) undirected = false;
            }
        }
        return sz;  // undirected ? sz / 2 : sz;
    }
\end{verbatim}%
}

The order of the graph is stored in an integer variable \verb|_order|.
However, we have to count all \verb|true| entries in the boolean
adjacency matrix to return the size.   Notice that if we  are working
with an undirected graph this returns twice the expected number
(since we store each edge as two arcs).   If we specialize this
class we may want to uncomment the indicated statements to autodetect
undirected graphs (whenever the matrix is symmetric). It is probably
safer to leave it as it is written, with the understanding that the
user knows how \emph{size} is defined for this implementation of
\verb|Graph|.


{\renewcommand{\ttdefault}{pcr} % for verbatim environment
\footnotesize \begin{verbatim}
  // default output is readable by constructor
  //
  public String toString() { return toStringAdjMatrix(); }
  
} // end class GraphAdjMatrix
\end{verbatim}%
}

We finish our implementation by setting our output method \verb|toString| 
to return an adjacency matrix.   Recall the method \verb|toStringAdjMatrix|
was presented earlier on page~\pageref{graphoutput}.

\section{Java adjacency lists implementation}

We now present an alternate implementation of our graph ADT using
the adjacency lists data structure.   We will use the Java API
class \verb|Vector| to store these lists.

{\renewcommand{\ttdefault}{pcr} % for verbatim environment
\footnotesize \begin{verbatim}
package graphADT;

import java.io.*;
import java.util.*;

/*  Current implementation uses adjacency lists form of a graph.
 */
public class GraphAdjLists implements Graph
{
    //  Internal Representation and Constructors
    //
    protected ArrayList<ArrayList<Integer>> adj;

    public GraphAdjLists()
    {
        adj = new ArrayList<ArrayList<Integer>>();
    }

    public GraphAdjLists(Graph G)
    {
        int n = G.order();
        adj = new ArrayList<ArrayList<Integer>>();
        for (int i = 0; i < n; i++)
        {
            adj.add(G.neighbours(i));
        }
    }
\end{verbatim}%
}

%We mimic the adjacency matrix code here with a private method to
%allocate memory. The method \verb|_allocate| creates a list of lists
%(that is, a \verb|Vector| of \verb|Vector|s). Note the created list, using
%\verb|new Vector()|, within the argument of the \verb|addElement| method.
We use an \verb|ArrayList| that contains an \verb|ArrayList| of
\verb|Integer| for our representation. We decided that the copy
constructor for \verb|Graph| is sufficient in terms of efficiency so do
not need to define a specialized copy constructor for
\verb|GraphAdjLists|, handled automatically by the Java runtime
environment.
The default constructor creates a list of no lists (that is, no vertices).
For better efficiency, the copy constructor takes over the role of our
allocator and appends the neighbour lists of the graph parameter $G$
directly onto a new adjacency list.

%\begin{note}
%Above we are probably illustrating a bad programming style.  Why?
%\end{note}


{\renewcommand{\ttdefault}{pcr} % for verbatim environment
\footnotesize \begin{verbatim}
    public GraphAdjLists(BufferedReader buffer)
    {
        try
        {
            String line = buffer.readLine().trim();
            String[] tokens = line.split("\\s+");

            if (tokens.length != 1)
            {
                throw new Error("bad format: number of vertices");
            }

            adj = new ArrayList<ArrayList<Integer>>();
            int n = Integer.parseInt(tokens[0]);

            for (int u = 0; u < n; u++)
            {
                ArrayList<Integer> current = new ArrayList<Integer>();
                line = buffer.readLine().trim();
                int limit = 0;
                if (!line.equals(""))
                {
                    tokens = line.split("\\s+");
                    limit = tokens.length;
                }

                for (int i = 0; i < limit; i++)
                {
                    current.add(Integer.parseInt(tokens[i]));
                }
                adj.add(current);
            }
        }
        catch (IOException x)
        { throw new Error("bad input stream"); }
    }
\end{verbatim}%
}

Our stream constructor reads in an integer denoting the order $n$ of the
graph and then reads in $n$ lines denoting the adjacency lists.  Notice
that we \emph{do not} check for correctness of the data.  For example,
a graph of 5 vertices could have erroneous adjacency lists with numbers
outside the range $0$ to $4$. We leave these robustness considerations
for an extended class to fulfil,  if desired.  Also note that we do not
list the vertex index in front of the individual lists and we use white
space to separate items. A blank line indicates an empty list (that is,
no neighbours) for a vertex.

{\renewcommand{\ttdefault}{pcr} % for verbatim environment
\footnotesize \begin{verbatim}
    //  Mutator Methods
    //
    public void addVertices(int n)
    {
        assert(0 <= n);
        if ( n > 0 )
        {
            for (int i = 0; i < n; i++)
            {
                adj.add(new ArrayList<Integer>());
            }
        }
    }

    public void removeVertex(int i)
    {
        assert(0 <= i && i < order());
        adj.remove(i);
        Integer I = new Integer(i);
        for (int u = 0; u < order(); u++)
        {
            ArrayList<Integer> current = adj.get(u);
            current.remove(I); // remove i from adj lists
            for (Integer num: current)
            {
                if (num > i)  // relabel larger indexed nodes
                {
                    int index = current.indexOf(num);
                    current.set(index, num-1);
                }
            }
        }
    }
\end{verbatim}%
}

Adding vertices is easy for our adjacency lists representation.   Here we
just expand the internal \verb|_adj| list by appending new empty lists.
The \verb|removeVertex| method is a little complicated in that we have
to scan each list to remove arcs pointing to the vertex being deleted.
We also have chosen to relabel vertices so that there are no gaps (that
is, we want vertex indexed by $i$ to be labeled \verb|Integer(i)|).  A good
question would be to find a more efficient \verb|removeVertex| method.
One way would be to also keep an in-neighbour list for each vertex.
However, the  extra data structure overhead is not desirable for our
simple implementation.

{\renewcommand{\ttdefault}{pcr} % for verbatim environment
\footnotesize \begin{verbatim}
    public void addArc(int i, int j)
    {
        assert(0 <= i && i < order());
        assert(0 <= j && j < order());
        if (isArc(i,j)) return;
        (adj.get(i)).add(j);
    }

    public void removeArc(int i, int j)
    {
        assert(0 <= i && i < order());
        assert(0 <= j && j < order());
        if (!isArc(i,j)) return;
        (adj.get(i)).remove(new Integer(j));
    }

    public void addEdge(int i, int j)
    {
        addArc(i,j);
        addArc(j,i);
    }

    public void removeEdge(int i, int j)
    {
        removeArc(i,j);
        removeArc(j,i);
    }
\end{verbatim}%
}

Adding and removing arcs is easy since the methods to do this 
exist in the \verb|Vector| class.  All we have to do 
is access the appropriate adjacency list.   We have decided
to place a safeguard in the \verb|addArc| method to prevent parallel
arcs from being added between two vertices.

{\renewcommand{\ttdefault}{pcr} % for verbatim environment
\footnotesize \begin{verbatim}
    //  Access Methods
    //
    public boolean isArc(int i, int j)
    {
        assert(0 <= i && i < order());
        assert(0 <= j && j < order());
        return (adj.get(i)).contains(new Integer(j));
    }

    public boolean isEdge(int i, int j)
    {
        return isArc(i,j) && isArc(j,i);
    }

    public int inDegree(int i)
    {
        assert(0 <= i && i < order());
        int sz = 0;
        for (int j = 0; j < order(); j++)
        {
            if (isArc(j,i))  sz++;
        }
        return sz;
    }

    public int degree(int i)
    {
        assert(0 <= i && i < order());
        return (adj.get(i)).size();
    }
\end{verbatim}%
}

Note how we assume that the \verb|contains| method of a \verb|Vector|
object does a data equality check and not just a reference check. The
\verb|outDegree| method probably runs in constant time since we just
return the list's size.   However, the \verb|inDegree| method has to
check all adjacency lists and could have to inspect  all arcs of the
graph/digraph.


{\renewcommand{\ttdefault}{pcr} % for verbatim environment
\footnotesize%
\begin{verbatim}
    public ArrayList<Integer> neighbours(int i)
    {
        assert(0 <= i && i < order());
        ArrayList<Integer> nei = new ArrayList<Integer>();
        for (Integer vert : adj.get(i))
        {
            nei.add(vert);
        }
        return nei;
        //return (ArrayList<Integer>)(adj.get(i)).clone();
    }

    public int order()
    {
        return adj.size();
    }

    public int size()    // Number of arcs (counts edges twice)
    {
        int sz = 0;
        for (int i=0; i<order(); i++)
        {
            sz += (adj.get(i)).size();
        }
        return sz;
    }
\end{verbatim}%
}

We do not want to have any internal references to the graph data structure
being available to non-class members.   Thus, we elected to return a clone
of the adjacency list for our \verb|neighbours| method. We did not want to
keep redundant data so the order of our graph is simply the size of the
\verb|adj| list. 

{\renewcommand{\ttdefault}{pcr} % for verbatim environment
\footnotesize \begin{verbatim}
// default output readable by constructor
//
public String toString() { return toStringAdjLists(); }

} // end class GraphAdjLists
\end{verbatim}%
}

Again, we have the default output format for this class be compatible 
with the constructor \verb|BufferedReader|.
(The method \verb|toStringAdjLists| is defined on
page~\pageref{graphoutput}.)

\section{Standardized Java graph class}

We now have two implementations of a graph class as
specified by our interface (abstract class) \verb|Graph|.  We want to write 
algorithms that can handle either format.  Since Java is object-oriented
we could have all our algorithms take a \verb|Graph| object and
the run-time dynamic mechanism should ascertain the correct adjacency
matrix or adjacency lists methods.
For example, we could write a graph coloring algorithm prototyped
as \verb|public int color(Graph G)|  and pass it
either a \verb|GraphAdjMatrix| or a \verb|GraphAdjLists|.
And it should work fine!\\ 

We next present a simple test program for how one would use our graph
implementations.  We encourage the reader to trace through the steps and
to try to obtain the same output.

{\renewcommand{\ttdefault}{pcr} % for verbatim environment
\footnotesize \begin{verbatim}
import java.io.*; import graphADT.*;

public class test {

public static void main(String argv[])
{
   Graph G1 = new GraphAdjLists();

   G1.addVertices(5);
   G1.addArc(0,2); G1.addArc(0,3); G1.addEdge(1,2);
   G1.addArc(2,3); G1.addArc(2,0); G1.addArc(2,4);
   G1.addArc(3,2); G1.addEdge(4,1); G1.addArc(4,2);

   System.out.println(G1);

   Graph G2 = new GraphAdjMatrix(G1);

   G2.removeArc(2,0); G2.removeArc(4,1); G2.removeArc(2,3);

   System.out.println(G2);

   Graph G3 = new GraphAdjLists(G2);

   G3.addVertices(2);
   G3.addArc(5,4); G3.addArc(5,2); G3.addArc(5,6);
   G3.addArc(2,6); G3.addArc(0,6); G3.addArc(6,0);

   System.out.println(G3);

   Graph G4 = new GraphAdjLists(G3);

   G4.removeVertex(4); G4.removeEdge(5,0); G4.addVertices(1);
   G4.addEdge(1,6);

   System.out.println(G4);
}
} // test
\end{verbatim}%
}

The expected output, using JDK version 1.6, is given in
Figure~\ref{fig:graphoutput}.   Note that the last version of the
digraph $G$ has a vertex of  out-degree zero in the adjacency lists. 
(To compile our program we type `javac test.java' and to execute it 
we type `java test' at our command-line prompt `\$'.)

\begin{figure}
\hspace*{.8in}\begin{minipage}{5in}
{\begin{verbatim}
$ javac test.java
$ java test
5
2 3 
2 4 
1 3 0 4 
2 
1 2 

5
0 0 1 1 0 
0 0 1 0 1 
0 1 0 0 1 
0 0 1 0 0 
0 0 1 0 0 

7
2 3 6 
2 4 
1 4 6 
2 
2 
4 2 6 
0 

7
2 3 
2 6 
1 5 
2 
2 5 

1 

\end{verbatim}%
}
\end{minipage}
\caption{Sample output of the graph \algfont{test} program.}
\label{fig:graphoutput}
\end{figure}
%\vfill

\section{Extended graph classes: weighted edges}
\label{sec:wgraphs}

The graph ADT presented in the previous sections can be easily extended
to provide a  customized data type.  For example, if one only wants
undirected graphs then a more restrictive class can be developed to
prevent arc operations.  In this section we want to illustrate how one
can develop an ADT for arc-weighted graphs. We first want to define a
new graph interface that allows for these weights.

{\renewcommand{\ttdefault}{pcr} % for verbatim environment
\footnotesize \begin{verbatim}
public interface wGraph extends Graph
{
    class Weight<X>
    {
        private X value;
        public Weight(X arg)
        {
            value = arg;
        }
        public X getValue()
        {
            return value;
        }
        public void setValue(X arg)
        {
            value = arg;
        }
        public String toString()
        {
            return value.toString();
        }
    }

    void addArc(int i, int j);  // overridden with "default weight"
    void addArc(int i, int j, Weight weight);   

    void setArcWeight(int i, int j, Weight weight);    
    //assumes edge i-j exists; and replaces the weight of edge i-j

    Weight getArcWeight(int i, int j);

    ArrayList<Weight> neighbourWeights(int i);   
    // If you call neighbours(i) and neighbourWeights(i) then 
    // the k-th element of both lists are correlated
}
\end{verbatim}%
}

In the above interface, we define a class to represent arbitrary
weights. Usually one uses \verb|Weight<Integer>| as the arc attributes.
Since \verb|wGraph| extends \verb|Graph|, the existing graph algorithms,
written for non-weighted graphs, should still work.

We can implement a \verb|wGraph| in several ways just like we had
\verb|GraphAdjMatrix| and \verb|GraphAdjLists| for the interface
\verb|Graph|.  We can then create a \verb|wGraph| object by creating an
implementation object, such as:\\
\verb|      wGraph wG = new wGraphMatrix(Buffer);|

An adjacency matrix implementation for this interface is given below.
Note that the \verb|BufferedReader| constructor assumes weights of type
\verb|Integer|. If one wants a floating point data type then another
constructor (or creation method) is required.

{\renewcommand{\ttdefault}{pcr} % for verbatim environment
\footnotesize \begin{verbatim}
package graphADT;

import java.util.ArrayList; import java.io.*;

public class wGraphMatrix implements wGraph
{
    protected int order;
    protected Weight[][] adjW;  // null entry means no arc

    public wGraphMatrix()
    {
       order = 0;
    }

    public wGraphMatrix(wGraphMatrix G)
    {
        int n = order = G.order();
        if ( n > 0 )
        {
            adjW = new Weight[n][n];
        }

        for (int i = 0; i < n; i++)
        {
            for (int j = 0; j < n; j++)
            {
                adjW[i][j] = G.adjW[i][j];
            }
        }
    }

    public wGraphMatrix(wGraph G) //convert implementation
    {
        int n = order = G.order();
        adjW = new Weight[n][n];

        for (int i = 0; i < n; i++)
        {
            ArrayList<Integer> nbrs = G.neighbours(i);
            ArrayList<Weight> wNbrs = G.neighbourWeights(i);
            for (int j = 0; j < nbrs.size(); j++)
            {
                int index = nbrs.get(j);
                adjW[i][index] = wNbrs.get(j);
            }
        }
    }

    public wGraphMatrix(Graph G) // promote and/or copy
    {
        int n = order = G.order();
        if ( n > 0 )
        {
            adjW = new Weight[n][n];
        }

        for (int i = 0; i < n; i++)
        {
            for (int j = 0; j < n; j++)
            {
                if (G.isArc(i, j))
                {
                    adjW[i][j] = new Weight<Integer>(1);
                }
            }
        }
    }

    public wGraphMatrix(BufferedReader buffer) 
    {
        try
        {
            String line = buffer.readLine().trim();
            String[] tokens = line.split("\\s+");

            if (tokens.length != 1)
            {
                throw new Error("bad format: number of vertices");
            }
            int n = order = Integer.parseInt(tokens[0]);

            if ( n > 0 )
            {
                adjW = new Weight[n][n];
            }

            for (int i = 0; i < n; i++)
            {
                line = buffer.readLine().trim();
                tokens = line.split("\\s+");
                if (tokens.length != n)
                {
                  throw new Error("bad format: adjacency matrix");
                }

                for (int j = 0; j < n; j++)
                {
                    int entry = Integer.parseInt(tokens[j]);
                    if (entry != 0)
                    {
                      adjW[i][j] = new Weight<Integer>(entry);
                    }
                }
            }
        }
        catch (IOException x)
        {
            throw new Error("bad input stream");
        }

    }

    // mutator methods

    public void addVertices(int n)
    {
        assert(0 <= n );
        Weight weights[][] = new Weight[order+n][order+n];

        for (int i = 0; i < order; i++)
        {
            for (int j = 0; j < order; j++)
            {
	            weights[i][j] = adjW[i][j];
            }
        }
        order += n;
        adjW = weights;
    }

    public void removeVertex(int v)
    {
        assert(0 <= v && v < order);
        order--;

        for (int i = 0; i < v; i++)
        {
            for (int j = v; j < order; j++)
            {
                adjW[i][j] = adjW[i][j+1];
            }
        }

        for (int i = v; i < order; i++)
        {
            for (int j = 0; j < v; j++)
            {
                adjW[i][j] = adjW[i+1][j];
            }
            for (int j = v; j < order; j++)
            {
                adjW[i][j] = adjW[i+1][j+1];
            }
        }
    }

    public void addArc(int i, int j)
    {
        assert(0 <= i && i < order());
        assert(0 <= j && j < order());
        adjW[i][j] = new Weight<Integer>(1); //default weight
    }

    public void removeArc(int i, int j)
    {
        assert(0 <= i && i < order());
        assert(0 <= j && j < order());
        adjW[i][j] = null;
    }

    public void addEdge(int i, int j)
    {
        addArc(i,j); addArc(j,i);
    }

    public void removeEdge(int i, int j)
    {
        removeArc(i,j); removeArc(j,i);
    }

    public void addArc(int i, int j, Weight weight)
    {
        assert(0 <= i && i < order());
        assert(0 <= j && j < order());
        adjW[i][j] = weight;
    }

    public void setArcWeight(int i, int j, Weight weight)
    {
        assert(isArc(i, j));
        adjW[i][j] = weight;
    }

    public Weight<?> getArcWeight(int i, int j)
    {
        assert(isArc(i, j));
        return adjW[i][j];
    }

    // accessor methods

    public boolean isArc(int i, int j)
    {
        assert(0 <= i && i < order);
        assert(0 <= j && j < order);
        return adjW[i][j] != null;
    }

    public boolean isEdge(int i, int j)
    {
        return isArc(i,j) && isArc(j,i);
    }

    public int inDegree(int i) // column count
    {
        assert(0 <= i && i < order);
        int sz = 0;
        for (int j = 0; j < order; j++)
        {
            if (adjW[j][i] != null) sz++;
        }
        return sz;
    }

    public int degree(int i) // row count
    {
        assert(0 <= i && i < order);
        int sz = 0;
        for (int j = 0; j < order; j++)
        {
            if (adjW[i][j] != null) sz++;
        }
        return sz;
    }

    public int order()
    {
        return order;
    }

    public int size()  // Number of arcs (edges count twice)
    {
        int sz = 0;
        for (int i = 0; i< order; i++)
        {
            for (int j = 0; j< order; j++)
            {
                if (adjW[i][j] != null) sz++;
            }
        }
        return sz;  // undirected ? sz / 2 : sz;
    }

    public ArrayList<Integer> neighbours(int i)
    {
        assert(0 <= i && i < order);
        ArrayList<Integer> nbrs = new ArrayList<Integer>();

        for (int j = 0; j < order; j++)
        {
            if (adjW[i][j] != null) nbrs.add(j); 
        }
        return nbrs;
    }

    public ArrayList<Weight> neighbourWeights(int i)
    {
        ArrayList<Weight> nbrsWei = new ArrayList<Weight>();

        for (int j = 0; j < order(); j++)
        {
            if (adjW[i][j] != null)
            {
               nbrsWei.add(adjW[i][j]); // corresponding weight
            }
        }
        return nbrsWei;
    }

    public String toString()  // print weights in n-by-n matrix 
    {
        StringBuffer o = new StringBuffer();
        o.append(order()+"\n");

        for (int i = 0; i < order(); i++)
        {
            for (int j = 0; j < order(); j++)
            {
                if (adjW[i][j] != null)
                {
                    o.append(adjW[i][j] + " ");
                }
                else
                {
                    o.append(0 + " ");
                }
            }
            o.append("\n");
        }
        return o.toString();
    }
 
}
\end{verbatim}%
}

One thing to note is that if one wants to output the underlying graph 
representation (that is, without weights) one can simply call 
the \verb|toString| method of \verb|Graph| reference.  \\

We conclude by mentioning that the details for an adjacency lists
implementation, \verb|wGraphLists|, are included in the graph library
accompanying this book.  We note that this adjacency lists version of the
ADT is more suitable when one expects weights of numerical value 0 or
has sparse graphs.

\if 01
The next program shows how to use both the \verb|wGraphMatrix|
and \verb|wGraphLists| classes.

{\renewcommand{\ttdefault}{pcr} % for verbatim environment
\footnotesize \begin{verbatim}
import java.io.*; import graphADT.*;

class myW extends wGraph.Weight<Integer> 
{ 
  myW(int x) { super(new Integer(x)); }
}

public class testwgraph {

public static void main(String argv[]) 
{
   wGraph G1 = new wGraphLists();

   G1.addVertices(5); 
   G1.addArc(0,2); G1.addArc(0,3); G1.addEdge(1,2); 
   G1.addArc(2,3); G1.addArc(2,0); G1.addArc(2,4); 
   G1.addArc(3,2); G1.addEdge(4,1); G1.addArc(4,2);

   System.out.println(G1);

   wGraph G2 = new wGraphMatrix(G1);

   G2.removeArc(2,0); G2.removeArc(4,1); G2.removeArc(2,3);

   G2.addVertices(1); G2.addArc(5,3); G2.addEdge(5,2);
   G2.removeVertex(5);

   System.out.println(G2);

   wGraph G3 = new wGraphLists(G2);

   G3.addVertices(2);
   G3.addArc(5,4); G3.addArc(5,2); G3.addArc(5,6);
   G3.addArc(2,6); G3.addArc(0,6); G3.addArc(6,0);

   System.out.println(G3);

   wGraph G4 = new wGraphLists(G3);

   G4.removeVertex(4); G4.removeEdge(5,0); G4.addVertices(1); 
   G4.addEdge(1,6);

   System.out.println(G4);

   wGraph GM = new wGraphMatrix();

   GM.addVertices(5); GM.addArc(0,2);
   GM.addArc(1,2,new myW(6));
   GM.addArc(3,2,new myW(7));
   GM.addArc(4,2,new myW(9));
   GM.addEdge(0,3); GM.setArcWeight(0,3,new myW(8));

   System.out.println(GM);
   System.out.println(new GraphAdjMatrix(GM));

   wGraph GL = new wGraphLists(GM);

   GL.removeVertex(2);
   GL.addArc(3,0,new myW(10));
   GL.setArcWeight(0,2,new myW(11));
   GL.addArc(3,2,new myW(12));
   GL.addEdge(1,0);
   GL.setArcWeight(1,0,new myW(13));

   System.out.println(GL);
   System.out.println(new GraphAdjLists(GL));
} 
} // end testwgraph
\end{verbatim}%
}

The output of this test program is given in Figure~\ref{fig:wgraphoutput}.


\begin{figure}
\hspace*{.8in}\begin{minipage}{5in}
{\begin{verbatim}
5
2 1 3 1 
2 1 4 1 
1 1 3 1 0 1 4 1 
2 1 
1 1 2 1 

5
0 0 1 1 0 
0 0 1 0 1 
0 1 0 0 1 
0 0 1 0 0 
0 0 1 0 0 

7
2 1 3 1 6 1 
2 1 4 1 
1 1 4 1 6 1 
2 1 
2 1 
4 1 2 1 6 1 
0 1 

7
2 1 3 1 
2 1 6 1 
1 1 5 1 
2 1 
2 1 5 1 

1 1 

5
0 0 1 8 0 
0 0 6 0 0 
0 0 0 0 0 
1 0 7 0 0 
0 0 9 0 0 

5
0 0 1 1 0 
0 0 1 0 0 
0 0 0 0 0 
1 0 1 0 0 
0 0 1 0 0 

4
2 11 1 1 
0 13 
0 1 
0 10 2 12 

4
2 1 
0 
0 
0 2 
\end{verbatim}%
}
\end{minipage}
\caption{Sample output of the weighted graph \algfont{test} program.}
\label{fig:wgraphoutput}
\end{figure}
\fi
