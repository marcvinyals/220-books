\chapter{Java code for Searching and Sorting}
\label{ch:app:sortsearchcode}

This appendix contains Java implementations for many of the
common search and sorting algorithms presented in 
the book.

\section{Sorting and selection}
\label{sec:app:order}

The Java class below contains class methods for sorting 
integer arrays and for selecting an array element of a given rank. 
These algorithms insertion sort, Shellsort, mergesort, quicksort, heapsort, 
and quickselect are described in detail in Chapter~\ref{CH:EFF:SORT}.  We
may place them all in a Java public class or they
may be cut-and-pasted, as needed, into a Java application.

{\footnotesize
\renewcommand{\ttdefault}{pcr} % for verbatim environment
\begin{verbatim}
   //  Insertion sort of an input array a of size n
   //
    public static void insertionSort( int [] a ) 
    { 
         for(  int i = 1; i < a.length; i++ ) { 
             int tmp = a [ i ]; 
             int k = i - 1; 
             while( k >= 0 && tmp < a[ k ] ) { 
                  a [k + 1 ] = a[ k ]; 
                  k--; 
             } 
             a[ k + 1 ] = tmp; 
         } 
   } 
 
   //  Selection sort of an input array a of size n
   //
   public static void selectionSort( int [] a ) 
   { 
         for( int i = 0; i < a.length - 1; i++ ) { 
             int posMin = i; 
             for( int k = i + 1; k < a.length; k++ ) {
                  if ( a[ posMin ] > a[ k ] ) posMin = k; 
             }
             if ( posMin != i ) swap( a, i, posMin );
         }
   } 
 
   //  Bubble sort of an input array a of size n
   //
   public static void bubbleSort( int [] a ) 
   { 
         for( int i = a.length - 1; i > 0; i-- ) { 
             for( int k = 0; k < i; k++ ) {
                  if ( a[ k ] > a[ k + 1 ] ) swap( a, k, k + 1 );
             }
        }
   } 
 
   //  Insertion sort of an input array a of size n: 
   //  a private method used by quicksort and quickselect: 
   //  sorting between the indices lo and hi: 0 <= lo <= hi < n
   //
   private static void insertionSort( int [] a, int lo, int hi ) 
   { 
        if ( lo > hi || lo < 0 || hi >= a.length ) { 
           lo = 0;
           hi = a.length - 1;
        } 
        for ( int i = lo + 1; i <= hi; i++ ) { 
             int tmp = a[i]; 
             int k = i - 1; 
             while( k >= lo && tmp < a[ k ] ) { 
                   a[ k + 1 ] = a[ k ]; 
                   k--; 
             } 
             a[ k + 1 ] = tmp; 
         } 
   } 

   //  Shell's sort of an input array a of size n
   //  using a sequence of gaps by G. Gonnet
   //
   public static void shellSort( int [] a ) 
   { 
       for( int gap = a.length/2; gap > 0; 
                    gap = (gap == 2) ? 1 : (int)(gap/2.2) ) 
            for( int i = gap; i < a.length; i++ ) { 
                int tmp = a[ i ]; 
                int k = i; 
                while( k >= gap && tmp < a [ k - gap ] ) { 
                     a[ k ] = a[ k - gap ]; 
                     k -= gap; 
                } 
               a[ k ] = tmp; 
          } 
   } 
 
   //  Mergesort of an input array a of size n
   //  using a temporary array to merge data
   //
   public static void mergeSort( int [] a) 
   { 
        int [] tmp = new int[ a.length ]; 
        mergeSort( a, tmp, 0, a.length - 1 ); 
   } 
 
   private static void mergeSort( int [] a, int [] tmp, 
                                  int left, int right) 
   { 
         if ( left < right ) { 
             int centre = (left + right) / 2; 
             mergeSort( a, tmp, left, centre); 
             mergeSort( a, tmp, centre + 1, right); 
             merge( a, tmp, left, centre + 1, right ); 
         } 
   } 
 
   private static void merge( int [] a, int [] tmp, 
                               int left, int right, int rend ) 
   { 
        int lend = right - 1; 
        int tpos = left; 
        int lbeg = left; 
     
        // Main loop 
        while( left <= lend && right <= rend ) 
              if ( a[ left ] < a[ right ] )
                 tmp[ tpos++ ] = a[ left++ ]; 
             else 
                tmp[ tpos++ ] = a[ right++ ]; 
 
        // Copy the rest of the first half 
       while( left <= lend ) 
             tmp[ tpos++ ] = a[ left++ ];
 
        // Copy the rest of the second half 
        while( right <= rend ) 
              tmp[ tpos++ ] = a[ right++ ]; 
 
        // Copy tmp array back 
        for( tpos = lbeg; tpos <= rend; tpos++ ) 
            a[ tpos ] = tmp[ tpos ]; 
   } 
 
   //  Quicksort of an input array a of size n  using a median-of-three pivot 
   //  and insertion sort of subarrays of size less than CUTOFF threshold:
   // 
   static public int CUTOFF = 10; 

   public static void quickSort( int [] a ) 
   { 
        quickSort( a, 0, a.length - 1 ); 
   } 

   private static void quickSort( int [] a, int lo, int hi ) 
   { 
        if ( lo + CUTOFF > hi ) 
           insertionSort( a, lo, hi ); 
       else { 
           // Sort low, middle, high 
           int mi = ( lo + hi ) / 2; 
           if ( a[ mi ] < a[ lo ] ) swap( a, lo, mi ); 
           if ( a[ hi ] < a[ lo ] ) swap( a, lo, hi ); 
           if ( a[ hi ] < a[ mi ] ) swap( a, mi, hi ); 
 
           // Place pivot p at position hi - 1 
           swap( a, mi, hi - 1 ); 
           int p = a[ hi - 1]; 
 
           // Begin partitioning 
          int i, j; 
          for ( i = lo, j = hi - 1; ; ) { 
               while( a[ ++i ] < p ); 
               while( p < a[ --j ] ); 
               if ( i < j ) swap( a, i, j ); 
               else break; 
           } 
 
           // Restore pivot 
           swap( a, i, hi - 1 ); 
           // Sort small elements
           quickSort( a, lo, i - 1 );
            // Sort large elements
           quickSort( a, i + 1, hi ); 
       } 
   } 
  
   private static void swap( int [] a, int i, int j ) 
   { 
     int tmp = a[ i ]; 
     a[ i ] = a[ j ]; 
     a[ j ] = tmp; 
   } 
 
   //  Heapsort of an input array a of size n
   //  using percolateDown() and swap() methods
   //
   public static void heapSort( int [] a ) 
   { 
        // build a heap 
        for ( int i = a.length / 2 - 1; i >= 0; i-- )
             percolateDown( a, i, a.length ); 
        // successively delete the max and restore the heap
        for( int i = a.length - 1; i >= 1; i-- ) { 
             swap( a, 0, i ); 
             percolateDown( a, 0, i ); 
        } 
    } 
	
    // Heapifying method to restore a heap a[0],...,a[size-1]
    // after changing a[ i ]; a child / parent position is one
    // greater than an index of the same array element
    private static void percolateDown( int [] a, int i, int size ) 
    { 
         int child;
         int parent = i + 1; 
     
         for ( child = parent * 2; child < size;  child = parent * 2 ) {	 
               if ( a[ parent - 1 ] < a[ child - 1 ] || 
                    a[ parent - 1 ] < a[ child ] ) {  
                  if ( a[ child - 1 ] < a[ child] ) { 
                     swap( a, parent - 1, child ); 
                     parent = child + 1; 
                  } else { 
                     swap( a, parent - 1, child - 1 ); 
                     parent = child; 
                  } 
              }  else break; 
         } 
         if ( child == size && a[ parent - 1 ] < a[ child - 1 ] ) 
            swap( a, parent - 1, child - 1 ); 
     } 
     
   //  Counting sort of an input array a of size n
   //  with elements such that min <= a[i] <= max 
   //  (assuming that this condition holds)
   //
   public static void countSort( int [] a, int min, int max ) 
   {
         int i, j;
         int m = max - min + 1;
         int [] accum = new int[ m ];
         for ( j = 0; j < m; j++ ) accum[ j ] = 0;
         for ( i = 0;  i < a.length; i++ ) accum[ a[ i ] ]++;
         for ( i = j = 0; j < m; j++ ) {
                if ( accum[ j ] == 0 ) continue;
                while( ( accum[ j ]-- ) > 0 )
                      a[ i++ ] = j + min;
         }     
   }
\end{verbatim}%
}% footnote

{\renewcommand{\ttdefault}{pcr} % for verbatim environment
\footnotesize 
\begin{verbatim}
    // Quick select in an input array a of size n: 
    // returns the k-th smallest element in a[ k - 1 ]
    //
    public static void quickSelect( int[] a, int k ) 
    { 
         quickSelect( a, 0, a.length - 1, k ); 
    } 
  
    private static void quickSelect( int[] a, int lo, int hi, int k ) 
    {
          if ( lo + CUTOFF > hi ) {
              insertionSort( a, lo, hi ); 
          } else { 
              // Sort low, middle, high
             int mi = ( lo + hi ) / 2; 
             if ( a[ mi ] < a[ lo ] ) swap( a, lo, mi ); 
             if ( a[ hi ] < a[ lo ] ) swap( a, lo, hi ); 
             if ( a[ hi ] < a[ mi ] ) swap( a, mi, hi ); 
 
              // Place the pivot p into the rightmost place 
              swap( a, mi, hi - 1 ); 
              int p = a[ hi - 1 ];

              // Begin partitioning 
              int i, j;
              for( i = lo, j = hi - 1; ; ) { 
                  while( a[ ++i ] < p ); 
                  while( p < a[ --j ] );
                  if ( i < j ) swap( a, i, j ); else break; 
              }  

              // Restore pivot  
              swap( a, i, hi - 1 ); 

              // Selection by recursion (the only changed part!) 
             if ( k - 1 < i ) quickSelect( a, lo, i - 1, k ); 
             else if ( k - 1 > i ) quickSelect( a, i + 1, hi, k ); 
         } 
     }   
\end{verbatim}%
}

%\newpage
\section{Search methods}

We now present several Java methods for searching 
in an integer array. The algorithms (sequential search and 
binary search) are described in detail 
in Chapter~\ref{CH:EFF:SEARCH}.

{\renewcommand{\ttdefault}{pcr} % for verbatim environment
\footnotesize \begin{verbatim}

  // Sequential search for key in an array a 
  //
  public static int sequentialSearch( int[] a, int key) 
  throws ItemNotFound 
  {
    for( int i = 0; i < a.length; i++ )
      if ( a[ i ] == key ) return i;
    throw new ItemNotFound( ``SequentialSearch fails'' );
  }	 

  // Binary search for key in a sorted array a
   // 
  public static int binarySearch( int[] a, int key) 
  throws ItemNotFound 
  {
     int lo = 0;
     int hi = a.length - 1;
     int mi;
    
     while( lo <= hi ) {
        mi = ( lo + hi ) / 2; 
        if ( a[ mi ] < key ) lo = mi + 1;
        else if( a[ mi ] > key ) hi = mi - 1;
        else return mi; 
     } 
     throw new ItemNotFound( "BinarySearch fails" );
  }	 

  // Binary search using two-way comparisons
  // 
  public static int binarySearch2( int[] a, int key) 
  throws ItemNotFound 
  {
    if ( a.length == 0 )
      throw new ItemNotFound( "Zero-length array" );

    int lo = 0;
    int hi = a.length - 1;
    int mi;
    
    while( lo < hi ) {
        mi = ( lo + hi ) / 2;
        if ( a[ mi ] < key ) lo = mi + 1;
        else                 hi = mi;
    }
    if ( a[ lo ] == key ) return lo;      
    throw new ItemNotFound( "BinarySearch fails" );
  }
\end{verbatim}%
}

%\newpage

\if 01 % not in e-book version

The Knuth-Morris-Pratt \algfont{KMP} algorithm, as described
in Section~\ref{sec:KMP}, for searching for a substring $X$ 
within a string $Y$ is now implemented using Java.

{\renewcommand{\ttdefault}{pcr} % for verbatim environment
\footnotesize \begin{verbatim}

void computeNext(String X, int[] next) 
{
   int i=0;
   int j=next[0]=-1;  // end of window marker

   while (i < X.length()) 
   {
      if (j > -1 && X[i] != X[j]) { j=next[j]; continue; }

      i++; j++;

      if (i==X.length()) { next[i]=j; break; }

      next[i]= X[i]==X[j] ? next[j] : j;
   }
}
\end{verbatim}

\begin{verbatim}
int KMP(String X, String Y) 
{
   int m=X.length(); 
   int n=Y.length();
   int[m+1] next; 

   computeNext(X,next);

   int i = 0; int j = 0; // indices in x and y
   while (j < n) 
   {
     while (i > -1 && X[i] != Y[j]) i=next[i];
     i++;
     j++;
     if (i >= m) return j-i; // Match
   }
   return -1; // Mismatch
}
\end{verbatim}%
}

\fi % not e-book version
