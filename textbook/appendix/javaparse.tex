\chapter{Recursive Descent Parsing}
\label{app:parsing}

\section{Templated parsing code for balanced parentheses grammar}

In this appendix we illustrate how to design a recursive descent parser 
for the following simple grammar.

\begin{eqnarray*}
  \X{B} & \rightarrow & \epsilon \\
  \X{B} & \rightarrow & ( \X{B} ) \: \X{B}\\
  \X{B} & \rightarrow & [ \X{C} ] \\
  \X{C} & \rightarrow & ( \X{B} ) , ( \X{B} )\\
\end{eqnarray*}

First we need to generate the overall recursive flow of the
parser using the templates given earlier in Section~\ref{sec:templateparsing}.
We need to define two methods \algfont{parseNonterminal}\texttt{B},
and 
\algfont{parseNonterminal}\texttt{C}, along with two auxiliary methods 
\algfont{parseProduction}\texttt{B1} and
\algfont{parseProduction}\texttt{B2} for the nonterminal $\X{B}$.  
These are given in Figure~\ref{fig:parsing}.
The reader may also want to review Example~\ref{ex:parsing}. 

\begin{figure}[p]
\hspace*{.8in}\begin{minipage}{5in}
\begin{tabbing}
xxxx\=xxxx\=xxxx\=xxxx\=xxxx\=xxxx\=xxxx\= \kill
\textbf{method} \algfont{parseNonterminal}\texttt{B}\\ 
\textbf{begin} \\
\> Token $T$ = \algfont{peek}()\\
\> \textbf{if} $T$==`(' \textbf{then}\\
\> \> \textbf{return} \algfont{parseProduction}\texttt{B1}\\
\> \textbf{elseif} $T$==`[' \textbf{then}\\
\> \> \textbf{return} \algfont{parseProduction}\texttt{B2}\\
\> \textbf{else} Check for $\X{B} \rightarrow \epsilon$ case.\\
\> \textbf{endif}\\
\textbf{end} 
\end{tabbing}

\begin{tabbing}
xxxx\=xxxx\=xxxx\=xxxx\=xxxx\=xxxx\=xxxx\= \kill
\textbf{method} \algfont{parseProduction}\texttt{B1} \\
\textbf{begin}\\
\> Tree $S$ = \textbf{new} Tree()\\
\> \algfont{check}('('); $S$.\algfont{addLeaf}('(')\\
\> $S$.\algfont{addSubtree}(\algfont{parseNonterminal}\texttt{B})\\
\> \algfont{check}(')'); $S$.\algfont{addLeaf}(')')\\
\> $S$.\algfont{addSubtree}(\algfont{parseNonterminal}\texttt{B})\\
\> \textbf{return} $S$\\
\textbf{end} 
\end{tabbing}

\begin{tabbing}
xxxx\=xxxx\=xxxx\=xxxx\=xxxx\=xxxx\=xxxx\= \kill
\textbf{method} \algfont{parseProduction}\texttt{B2} \\
\textbf{begin}\\
\> Tree $S$ = \textbf{new} Tree()\\
\> \algfont{check}('['); $S$.\algfont{addLeaf}('[')\\
\> $S$.\algfont{addSubtree}(\algfont{parseNonterminal}\texttt{C})\\
\> \algfont{check}(']'); $S$.\algfont{addLeaf}(']')\\
\> \textbf{return} $S$\\
\textbf{end} 
\end{tabbing}

\begin{tabbing}
xxxx\=xxxx\=xxxx\=xxxx\=xxxx\=xxxx\=xxxx\= \kill
\textbf{method} \algfont{parseNonterminal}\texttt{C}\\ 
\textbf{begin} \\
\> Tree $S$ = \textbf{new} Tree()\\
\> \algfont{check}('('); $S$.\algfont{addLeaf}('(')\\
\> $S$.addSubtree(\algfont{parseNonterminal}\texttt{B})\\
\> \algfont{check}(')'); $S$.\algfont{addLeaf}(')')\\
\> \algfont{check}(','); $S$.\algfont{addLeaf}(',')\\
\> \algfont{check}('('); $S$.\algfont{addLeaf}('(')\\
\> $S$.\algfont{addSubtree}(\algfont{parseNonterminal}\texttt{B})\\
\> \algfont{check}(')'); $S$.\algfont{addLeaf}(')')\\
\textbf{end} 
\end{tabbing}
\end{minipage}
\caption{A recursive descent parser for a simple grammar.} \label{fig:parsing}
\end{figure}

\section{Java implementation for balanced parentheses grammar}

We now want to convert the template parsing code of the previous section
into a working Java program.  We first need to define the stream input
and parse tree helper classes.  Our simple tokenizer for this example
will be the builtin Java \verb|StringTokenizer|, which will break
up the input at white spaces.  

In the code that follows 
we write our \verb|check| method to return a \verb|boolean| value
instead of throwing an exception so that we can easily illustrate
the location of errors in the bad input strings.  

%\newpage
%The Java
%implementation begins on page~\pageref{parsecode}.
%\newpage
\label{parsecode}
{\footnotesize%
\renewcommand{\ttdefault}{pcr} % for verbatim environment
\begin{verbatim}
import java.util.StringTokenizer;

public class ParseInput 
{ 
  String str, curToken;
  boolean haveCurToken;
  StringTokenizer tokens;

  public ParseInput( String s ) 
  { 
    str = new String(s);  
    haveCurToken = false;
    tokens = new StringTokenizer(str);
  }

  String peek()
  {
    if (haveCurToken) return curToken;
    else
    {
      if (tokens.hasMoreTokens())
      { 
         curToken = tokens.nextToken();  
         haveCurToken=true;
         return curToken;
      }
      else return new String();  // end of input indicator
    }
  }

  boolean check(String s)
  {
    if (haveCurToken) 
    { 
      haveCurToken=false; 
      return s.equals( curToken ); 
    }
    else 
    { 
      if (tokens.hasMoreTokens()) return s.equals( tokens.nextToken() ); 
      else return false;
    }
  }

  boolean checkEndInput()
  {
    if (haveCurToken) return false;
    else return tokens.hasMoreTokens()==false; 
  }
}
\end{verbatim}%
}%\footnotesize

Our data structure for the returned parse trees will be an ordered
rooted tree of strings.  The methods \verb|addLeaf| and 
\verb|addSubtree| do what is expected by appending children 
to an \verb|ArrayList| of siblings.
The only complicated part of our implementation is the \verb|toString| 
method which outputs the parse tree in a readable format, where we
indent to the right for each depth level of the parse tree.
Lines printed with the same indentation (top-to-bottom order) correspond
to the left-to-right ordering of the siblings.

%\newpage
{\footnotesize% 
\renewcommand{\ttdefault}{pcr} % for verbatim environment
\begin{verbatim}
import java.util.*;

public class PTree
{ 
  String obj;
  ArrayList subtrees;
  
  public PTree(String root) 
  { 
    obj = new String(root);
    subtrees = new ArrayList();
  }

  public void addLeaf(String leaf)
  {
    subtrees.add(new PTree(leaf)); // with empty subtree
  }

  public void addSubtree(PTree subtree)
  {
    subtrees.add(subtree);  // note: reference to subtree
  }

  static int tabpos=0;
  //
  public String toString()
  {
    StringBuffer sb = new StringBuffer(obj);

    if (subtrees.size()>0) 
    {
       tabpos += obj.length()+3;
       sb.append("-->");
       for (int i=0; i<subtrees.size(); i++)
       {
          if (i>0) for (int j=0; j<tabpos; j++) sb.append(" ");
          sb.append( subtrees.get(i).toString() );
          if (i<subtrees.size()-1) sb.append("\n");
       }
       tabpos -= obj.length()+3;
    }

    return sb.toString();
  }
}
\end{verbatim}%
}%\footnotesize

Next we write Java code for the recursive descent parser.  Notice the
close correspondence between this and 
to the pseudo-code of the previous section.  To make things slightly
more sophisticated we add error strings to the constructed parse tree, 
whenever our method \verb|check| does not match an input token.

{\footnotesize%
\renewcommand{\ttdefault}{pcr} % for verbatim environment
\begin{verbatim}
import java.io.*;
import java.lang.*;

public class ParseBP 
{ 
  static ParseInput in;

  static public PTree ParseBP(String s) // mainParser
  {
    in = new ParseInput(s);  // start token input stream from s

    PTree PT = ParseNonterminalB();
    if ( ! in.checkEndInput() ) PT.addLeaf("errorEOI"); 
    return PT;  
  }

  static public PTree ParseNonterminalB()
  {
    String T=in.peek();

    // match epsilon or non-processed )
    //
    if (T.length()==0 || T.equals(")"))  
    { 
        return new PTree("B-->null"); 
    }
    if (T.equals("(")) return ParseProductionB1();
    if (T.equals("[")) return ParseProductionB2(); 
    return new PTree("errorChar"); 
  }

  static public PTree ParseProductionB1()
  {
    PTree S = new PTree("B1");
    if (in.check("(")) S.addLeaf("("); else S.addLeaf("error(");
    S.addSubtree(ParseNonterminalB());
    if (in.check(")")) S.addLeaf(")"); else S.addLeaf("error)");
    S.addSubtree(ParseNonterminalB());
    return S;
  }

  static public PTree ParseProductionB2()
  {
    PTree S = new PTree("B2");
    if (in.check("[")) S.addLeaf("["); else S.addLeaf("error[");
    S.addSubtree(ParseNonterminalC());
    if (in.check("]")) S.addLeaf("]"); else S.addLeaf("error]");
    return S;
  }

  static public PTree ParseNonterminalC()
  {
    PTree S = new PTree("C");
    if (in.check("(")) S.addLeaf("("); else S.addLeaf("error(");
    S.addSubtree(ParseNonterminalB());
    if (in.check(")")) S.addLeaf(")"); else S.addLeaf("error)");
    if (in.check(",")) S.addLeaf(","); else S.addLeaf("error,");
    if (in.check("(")) S.addLeaf("("); else S.addLeaf("error(");
    S.addSubtree(ParseNonterminalB());
    if (in.check(")")) S.addLeaf(")"); else S.addLeaf("error)");
    return S;
  }
}
\end{verbatim}%
}% footnotesize

One thing to observe in the above method
\algfont{parseNonterminal}\texttt{B}
is that the grammar is designed so that we do not need to check for `]' 
when matching the $\X{B}\rightarrow \epsilon$ case.

%\newpage
Finally, in the same class \verb|ParseBP| that is given above (note the
\verb|static| modifiers) we write a 
main method to do some testing on some input strings of whitespace
separated tokens.
 
{\footnotesize% 
\renewcommand{\ttdefault}{pcr} % for verbatim environment
\begin{verbatim}
public static void main( String[] args ) 
{
  // testing simple case
  System.out.println( "Input 1: ( )" );
  System.out.println( ParseBP( "( )" ) );

  // testing nested B
  System.out.println( "Input 2: ( ( ) ( ) )" );
  System.out.println( ParseBP( "( ( ) ( ) )" ) );

  // testing simple B followed by C
  System.out.println( "Input 3: ( ) [ ( ) , ( ) ]" );
  System.out.println( ParseBP( "( ) [ ( ) , ( ) ]" ) );

  // error since things after the last bracket ]
  System.out.println( "Input 4: [ ( ( ) ( ) ) , ( ) ] ( )" );
  System.out.println( ParseBP( "[ ( ( ) ( ) ) , ( ) ] ( )" ) );

  // testing nested nonterminal C
  System.out.println( "Input 5: [ ( ( ) [ ( ) , ( ) ] ) , ( ) ]");
  System.out.println( ParseBP( "[ ( ( ) [ ( ) , ( ) ] ) , ( ) ]" ) );

  // error with bad character
  System.out.println( "Input 6: ( ( x ) )"); 
  System.out.println( ParseBP( "( ( x ) )" ) ); 

  // error inside evaluation of nonterminal C
  System.out.println( "Input 7: ( ) [ ( ) , ]"); 
  System.out.println( ParseBP( "( ) [ ( ) , ]" ) ); 
}
\end{verbatim}%
}%footnotesize

We show the output of our special parentheses parser on some input examples
in Figure~\ref{fig:parseoutput}.

\begin{figure}[p]
{\footnotesize
%\hspace*{-.1in}
\begin{tabular}{ll}
\begin{minipage}{3in}
\begin{verbatim}
Input 1: ( )
B1-->(
     B-->null
     )
     B-->null

Input 2: ( ( ) ( ) )
B1-->(
     B1-->(
          B-->null
          )
          B1-->(
               B-->null
               )
               B-->null
     )
     B-->null

Input 3: ( ) [ ( ) , ( ) ]
B1-->(
     B-->null
     )
     B2-->[
          C-->(
              B-->null
              )
              ,
              (
              B-->null
              )
          ]

Input 4: [ ( ( ) ( ) ) , ( ) ] ( )
B2-->[
     C-->(
         B1-->(
              B-->null
              )
              B1-->(
                   B-->null
                   )
                   B-->null
         )
         ,
         (
         B-->null
         )
     ]
     errorEOI
\end{verbatim}
\end{minipage} & \hspace*{-4ex}
\begin{minipage}{3.5in}
\begin{verbatim}
Input 5: [ ( ( ) [ ( ) , ( ) ] ) , ( ) ]
B2-->[
     C-->(
         B1-->(
              B-->null
              )
              B2-->[
                   C-->(
                       B-->null
                       )
                       ,
                       (
                       B-->null
                       )
                   ]
         )
         ,
         (
         B-->null
         )
     ]


Input 6: ( ( x ) )
B1-->(
     B1-->(
          errorChar
          error)
          B-->null
     )
     B-->null
     errorEOI


Input 7: ( ) [ ( ) , ]
B1-->(
     B-->null
     )
     B2-->[
          C-->(
              B-->null
              )
              ,
              error(
              B-->null
              error)
          error]
\end{verbatim} 
\end{minipage} 
\end{tabular} }
\caption{Sample parse trees obtained from our program \algfont{parseBP}.}
\label{fig:parseoutput}
\end{figure}
