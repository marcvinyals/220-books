\chapter{The Graph Abstract Data Type}
\label{ch:graphadt}

Many real-world applications, such as building compilers, finding routing
protocols in communication networks, and scheduling processes, are well
modelled using concepts from graph theory. A (directed) graph is an
abstract mathematical model:  a set of points and connection relations
between them.

\section{Basic definitions}
\label{sec:graphdefs}

Graphs can be studied from a purely mathematical point of view (``does
something exist, can something be done?"). In this book we focus on
algorithmic aspects of graph theory (``how do we do it efficiently and
systematically?").  However, the mathematical side is where we must
start, with precise definitions. The intuition behind the definitions is
not hard to guess.

We start with the concept of digraph (the word stands for
\textbf{di}rected \textbf{graph}). A good example to keep in mind 
is a street network with one-way streets. 

\begin{Definition}\label{def:digraph} 

A \defnfont{digraph} $G=(V,E)$ is a  finite nonempty set $V$ of \defnfont{nodes}
together with a (possibly empty) set $E$ of ordered pairs of nodes of
$G$ called \defnfont{arcs}.

\end{Definition}

\begin{note}   
\

\begin{itemize} 
\item 
In the mathematical language of relations,
the definition says that $E$ is a relation on $V$. If $(u,
v)\in E$, we say that $v$ is \defnfont{adjacent} to $u$, that
$v$ is an \defnfont{out-neighbour} of $u$, and that $u$ is an
\defnfont{in-neighbour} of $v$.
\item
We can think of a node as being a point and an arc as an arrow from one 
node to another. This allows us to draw pictures that suggest ideas. The 
pictures cannot \boldfont{prove} anything, however.
\end{itemize}
\end{note}

Very often the adjacency relation is symmetric (all streets are
two-way). There are two ways to deal with this. We can use a digraph
that happens to be symmetric (in other words, $(u, v)$ is an arc if and
only if $(v, u)$ is an arc). However, it is sometimes simpler to reduce
this pair of arcs into a single undirected edge that can be traversed in
either direction.

\begin{Definition}\label{def:graph}
A \defnfont{graph} $G=(V,E)$ is a finite nonempty  set $V$ of 
\defnfont{vertices} together with a (possibly empty) set $E$ of unordered
pairs of vertices of $G$ called \defnfont{edges}.
\end{Definition}

\begin{note}
\

\begin{itemize}
\item
Since we defined $E$ to be a set, there are no multiple arcs/edges
between a given pair of nodes/vertices.
\item
Non-fluent speakers of English please note: the singular of ``vertices" is not
``vertice", but ``vertex".
\item
For a given digraph $G$ we may also denote the set of nodes by $V(G)$
and the set of arcs by $E(G)$ to lessen any ambiguity.
\end{itemize}
\end{note}

\begin{Example}
\label{eg:graphExample}

We display a graph $G_1$ and a digraph $G_2$ in
Figure~\ref{fig:graphExample}. The
nodes/vertices are labelled $0, 1, \dots $ as in the picture. The arcs
and edges are as follows.

$$E(G_1) = \set{\{0,1\},\{0,2\},\{1,2\},\{2,3\},\{2,4\},\{3,4\}}$$

$$E(G_2) = \set{(0,2),(1,0),(1,2),(1,3),(3,1),(3,4),(4,2)}$$


\begin{figure}
\centerline{\Ipe{./figs/graphEx.ipe}}
\caption{A graph $G_1$ and a digraph $G_2$.}
\label{fig:graphExample}
\end{figure}

\end{Example}

\begin{note} 
Some people like to view a graph as a special type of digraph where
every unordered edge ${u,v}$ is replaced by two directed arcs $(u,v)$
and $(v,u)$.  This has the advantage of allowing us to consider only
digraphs, and we shall use this approach in our Java implementation in 
Appendix~\ref{app:javagraph}. It works in most instances.

However, there are disadvantages; for some purposes we must know whether
our object is really a graph or just a symmetric digraph. Whenever there
is (in our opinion) a potential ambiguity, we shall point it out.
\end{note}

\begin{Example}
Every  rooted tree can be interpreted as a digraph: there is an arc from
each node to each of its children. 
\end{Example}

\begin{note}(Graph terminology)
\begin{itemize}
\item
The terminology in this subject is unfortunately not completely
standard. Some authors call a graph by the longer term ``undirected
graph'' and use the term ``graph" to mean what we call a directed graph.
However when using our definition of a graph,  it is
standard practice to abbreviate the phrase ``directed graph'' with the word digraph.
\item
We shall be dealing with both graphs and digraphs throughout these
notes. In order to save writing ``(di)graph" too many times, we make the
following convention.  We treat the digraph as the fundamental concept.
In other words, we shall use the terminology of digraphs, nodes and
arcs, with the understanding that if this is changed to graphs, edges,
and vertices, the resulting statement is still true. However, if we talk
about graphs, edges, and vertices, our statement is not necessarily true
for digraphs. Whenever a result is true for digraphs but not for graphs,
we shall say this explicitly (this happens very rarely).
\item
There is another convention to discuss. An arc that begins and ends at
the same node is called a \defnfont{loop}. We make the convention that
\textbf{loops are not allowed in our digraphs}. Again, other authors may
differ. If our conventions are relaxed to allow multiple arcs and/or
loops, many of the algorithms below work with no modification or with
only very minor modification required. However dealing with loops
frequently requires special cases to be considered, and  would distract
us from our main goal of introducing the field of graph algorithms. As
an example of the problems caused by loops, suppose that we represent a
graph as a symmetric digraph as described above. How do we represent a loop in
 the graph?
\end{itemize}
\end{note}


\begin{Definition} 
The \defnfont{order} of a digraph $G=(V,E)$ is $|V|$, the number of nodes. 
The \defnfont{size} of $G$ is $|E|$, the number of arcs.

\end{Definition}

\begin{note} 
\

\begin{itemize}
\item
We shall usually use $n$ to denote $|V|$ and $e$ to denote
$|E|$.
\item
For a given $n$, the value of $e$ can be as low as $0$ (a digraph
consisting of $n$ totally disconnected points)  and as high as $n(n-1)$
(each node can point to each other node; recall that we do not allow
loops). If $e$ is toward the low end, the digraph is called \defnfont{sparse}, and 
if $e$ is toward the high end, then the digraph is called \defnfont{dense}. These
terms are obviously very informal. For our purposes we will call a class of 
digraphs sparse if $e$ is $O(n)$ and dense if $e$ is $\Theta(n^2)$.
\end{itemize}
\end{note}

\begin{Definition} 
A \defnfont{walk} in a digraph $G$ is a sequence of nodes $v_0\, v_1\,
\ldots\, v_n$ such that, for each $i$ with $0 \leq i < n$, $(v_i,
v_{i+1})$ is an arc in $G$. The \defnfont{length} of the walk $v_0\, v_1\,
\ldots \,v_n$ is the number $n$ (that is, the number of arcs involved).

A \defnfont{path} is a walk in which no node is repeated. A (simple) \defnfont{cycle} 
is a walk in which $v_0 = v_n$ and no arc (or edge for undirected case) 
is repeated. 
\end{Definition}

\begin{note}
\

\begin{itemize}
\item The definition of cycle is fine for digraphs but must be changed
slightly for graphs. We rule out a walk of the form $u\, v\, u$ as a
cycle (going back and forth along the same edge shouldn't count as a
cycle). Thus a cycle in a graph must be of length at least $3$.
\item It is easy to see that if there is a walk from $u$ to $v$, then
(by omitting some steps) we can find a path from $u$ to $v$.
\end{itemize}
\end{note}

\begin{Example}
For the graph $G_1$ of Example~\ref{eg:graphExample} the following
sequences of vertices are classified as being walks, paths, or cycles.

\medskip

\begin{center}
\begin{tabular}{|l|c|c|c|}\hline
vertex sequence & walk? & path? &  cycle? \\ \hline
$0\, 3\, 2$ & no & no & no  \\
$0\, 1\, 2\, 3\, 4$ & yes & yes & no \\
$0\, 1\,  2\,  0$ & yes & no & yes  \\
$0\,  1\,  2\,  3\,  4\,  2\,  0$ & yes & no & no \\
$0 \, 1\,  0$ & yes & no & no \\
\hline
\end{tabular}
\end{center}
\end{Example}

\begin{Example}
For the digraph $G_2$ of Example~\ref{eg:graphExample} the following
sequences of nodes are classified as being walks, paths, or cycles.
\medskip

\begin{center}
\begin{tabular}{|l|c|c|c|}\hline
node sequence & walk? & path? & cycle?  \\ \hline
$0\,  1\,  2\,  3\,  4$ & no & no & no  \\
$0\,  2\,  4$ & no & no & no \\
$3\,  1\,  2$ & yes & yes & no \\
$1\,  3\,  1$ & yes & no & yes  \\
$3\,  1\,  3\,  1$ & yes & no & no \\ 
\hline
\end{tabular}
\end{center}
\end{Example}

\begin{Definition} 
In a graph, the \defnfont{degree} of a vertex $v$ is the number of edges
meeting $v$. In a digraph, the \defnfont{outdegree} of a node $v$ is the
number of arcs leaving $v$, and the \defnfont{indegree} of $v$ is the
number of arcs entering $v$.

A node of indegree $0$ is called a \defnfont{source} and a node of 
outdegree $0$ is called a \defnfont{sink}.
\end{Definition}

If the nodes have a natural order, we may simply list the indegrees or
outdegrees in a sequence.

\begin{Example}
For our graph $G_1$, we have the degree sequence is $(2, 2, 4, 2, 2)$.
The indegree sequence and outdegree sequence of the digraph $G_2$ are
$(1,1,3,1,1)$ and $(1,3,0,2,1)$, respectively. Node $2$ is a sink.
\end{Example}

\begin{Definition}
The \defnfont{distance} from $u$ to $v$ in $G$, denoted by $d(u,v)$, is the 
minimum length of a path from $u$ to $v$. If no path exists, the distance is 
undefined (or $+\infty$).
\end{Definition}

For graphs, we have $d(u,v) = d(v,u)$ for all vertices $u, v$. 

\begin{Example}
In graph $G_1$ of Example~\ref{eg:graphExample}, we can see by considering
all possibilities that $d(0, 1) = 1$, $d(0, 2) = 1$, $d(0, 3) = 2$,
$d(0, 4) = 2$, $d(1, 2) = 1$, $d(1, 3) = 2$, $d(1, 4) = 2$, $d(2, 3) =
1$, $d(2, 4) = 1$ and $d(3, 4) = 1$.

In digraph $G_2$ of that Example, we have, for example, $d(0, 2) = 1, 
d(3, 1) = 2$. Since node $2$ is a sink, $d(2, v)$ is not defined 
unless $v = 2$, in which case the value is $0$.
\end{Example}

There are several ways to create new digraphs from old ones.

One is to delete (possibly zero) nodes and arcs in such a way that the 
resulting object is still a digraph (there are no arcs missing any
endpoints!).

\begin{Definition}
A \defnfont{subdigraph} of a digraph $G = (V, E)$ is a digraph $G' = (V',
E')$ where $V'\subseteq V$ and $E'\subseteq E$. A \defnfont{spanning}
subdigraph is one with $V'=V$; that is, it contains all nodes.

\end{Definition}

\begin{Example}

\end{Example}

\begin{Definition}
The subdigraph \defnfont{induced} by a a subset $V'$ of $V$ is the 
digraph $G' = (V', E')$ where $E' = \set{(u, v) \in E \mid u \in V', v\in V'}$.
\end{Definition}

\begin{Example}

\end{Example}

We shall sometimes find it useful to "reverse all the arrows''.

\begin{Definition}
The \defnfont{reverse} of the digraph $G = (V, E)$, is the 
digraph $G_r = (V, E')$ where $(u, v)\in E'$ if and only if $(v, u)\in E$.
\end{Definition}

\begin{Example}


\end{Example}

It is sometimes useful to replace all one-way streets with two-way
streets. The formal definition must take care not to introduce multiple
edges. Note below that if $(u, v)$ and $(v, u)$ belong to $E$, then only
one edge joins $u$ and $v$ in $G'$.  This is because $\{u, v\}$ and
$\{v, u\}$ are equal as sets, so appear only once in the set $E'$.

\begin{Definition}
The \defnfont{underlying graph} of a digraph $G = (V, E)$ is the graph 
$G' = (V, E')$ where $E' = \set{\{u, v\} \mid (u, v)\in E}$.

\end{Definition}

\begin{Example}


\end{Example}


We may need to combine two or more graphs $G_1, G_2$, \ldots $G_k$ into a single
graph where the vertices of each $G_i$ are completely disjoint from each other.  
The constructed graph $G$ is call the \defnfont{graph union}, 
where $V(G) = V(G_1) \cup V(G_2) \cup \ldots \cup V(G_k)$ and $E(G) = E(G_1) \cup
E(G_2) \cup \ldots \cup E(G_k)$.

\begin{Exercise}
\label{ex:degree}
Prove that in a digraph, the sum of all outdegrees equals the sum of all 
indegrees. What is the analogous statement for a graph? 
\end{Exercise}

\begin{Exercise}
\label{ex:distbound}
Let $G$ be a digraph of order $n$ and $u, v$ nodes of $G$. 
Show that $d(u, v) \leq n - 1$.
\end{Exercise}

\begin{Exercise}
\label{ex:sparse-deg}
Prove that in a sparse digraph, the average indegree of a node is $O(1)$, 
while in a dense digraph, the  average indegree of a node is $\Theta(n)$.
\end{Exercise}

\section{Digraphs and data structures}\label{sec:graph-reps}

In order to process digraphs by computer we first need to consider how
to represent them in terms of data structures. There are two common
computer representations for digraphs, which we now present. We assume
that the digraph has the nodes given in a fixed order. \boldfont{Our
convention is that the vertices are labelled $0, 1, \dots, n - 1$.}

\begin{Definition}

Let $G$ be a digraph of order $n$. The \defnfont{adjacency matrix} of $G$
is the $n\times n$ boolean matrix (often encoded with $0$'s and $1$'s)
such that entry $(i,j)$ is true if and only if there is an arc from the
node $i$ to node $j$.

\end{Definition}

\begin{Definition}
For a digraph $G$ of order $n$, an \defnfont{adjacency lists}
representation is a sequence of $n$ lists, $L_0, \dots, L_{n-1}$. List $L_i$  
contains all nodes of $G$ that are adjacent to node $i$.
\end{Definition}

We can see the structure of these representations more clearly with
examples.

\begin{Example}

For the graph $G_1$ and digraph $G_2$ of Example~\ref{eg:graphExample}, the 
adjacency matrices are given below.

$$
G_1: 
\left[
\begin{matrix}
0 & 1 & 1 & 0 & 0 \\
1 & 0 & 1 & 0 & 0 \\
1 & 1 & 0 & 1 & 1 \\
0 & 0 & 1 & 0 & 1 \\
0 & 0 & 1 & 1 & 0 
\end{matrix}
\right]
\qquad 
G_2: 
\left[
\begin{matrix}
0 & 0 & 1 & 0 & 0 \\
1 & 0 & 1 & 1 & 0 \\
0 & 0 & 0 & 0 & 0 \\
0 & 1 & 0 & 0 & 1 \\
0 & 0 & 1 & 0 & 0 
\end{matrix}
\right]
$$

Notice that the number of $1$'s in a row (column) is the outdegree
(indegree) of the corresponding node. The corresponding adjacency lists are given below.

\begin{center}
$$
G_1: \quad
\begin{array}{cccc}
1 & 2  \\
0 & 2 \\
0 & 1 & 3 & 4  \\
2 & 4  \\
2 & 3  \\
\end{array}
\qquad
G_2: 
\quad 
\begin{array}{ccc}
2  \\
0 & 2 & 3  \\
\\
1 & 4 \\
2 \\
\end{array}
$$
\end{center}

\end{Example}

\begin{note}
Only the out-neighbours are listed in the adjacency lists
representation.   An empty list can occur
(for example, list $2$ of the digraph $G_2$). If the nodes are not numbered in the usual way (for example, they are numbered $1, \dots , n$ or labelled $A, B, C, \dots$), we may include these labels if necessary.
\end{note}

It is often useful to input several digraphs from a single file. Our
standard format is as follows. The file consists of several digraphs 
one after the other. To distinguish the beginning of one and the end of
the other we have a single line giving the order at the beginning of
each graph. If the order is $n$ then the next $n$ lines give the
adjacency matrix or adjacency lists representation of the digraph. 
The end of the file is marked with a line containing  the single character 
$0$.

\begin{Example}

Here is a file consisting of 3 digraphs, of orders 4, 3, 1 respectively.

$$
\begin{tabular}{lll}
4 & & \\
1 & 2 & \\
2 & 0 &\\
3 & &\\
& & \\
3 &  & \\
2 & &\\
0 &  &  \\
1 &  & \\
1 & & \\
 & & \\
0 & & \\
\end{tabular}
$$

\end{Example}


There are also other specialized (di)graph representations besides the
two mentioned in this section.  These data structures take advantage of
special structure for improved storage or access time, often for
families of graphs sharing a common property. For such specialized
purposes they may be better than either the adjacency matrix or lists
representations.

For example, trees can be stored much more efficiently. We have already
seen in Section~\ref{sec:heapsort} how a complete binary tree can be
stored in an array. A general rooted tree of $n$ nodes can be stored in
an array $pred$ of size $n$. The value $pred[i]$ gives the parent of
node $i$. The root is a special case and can be given value $-1$
(representing a NULL pointer), for example, if we number nodes from $0$
to $n-1$ in the usual way. This of course is a form of adjacency lists
representation, where we use in-neighbours instead of out-neighbours.

We will sometimes need to represent $\infty$ when processing graphs. For
example, it may be more convenient to define $d(u, v) = \infty$ than to
say it is undefined. From a programming point of view, we can use any
positive integer that can not be confused with any other that might
legitimately arise. For example, the distance between 2 nodes in a
digraph on $n$ nodes cannot be more than $n - 1$ (see
Exercise~\ref{ex:distbound}). Thus in this case we may use $n$ to
represent the fact that there is no path between a given pair of nodes.
We shall return to this subject in Chapter~\ref{ch:weighted}.

\begin{Exercise}
\label{ex:list2matrix}

Write down the adjacency matrix of the digraph whose adjacency lists
representation is given below.
\newline
$$
\begin{tabular}{cccc}
2 &  & \\
0 & &\\
0 & 1&\\
4 & 5 & 6\\
5 & &\\
3 & 4 & 6 \\
1 & 2 & \\
\end{tabular}
$$

\end{Exercise}

\begin{Exercise}[2003FT test]
\label{ex:matrix2list}

Consider the digraph $G$ with nodes $0, \dots ,6$ whose adjacency matrix
representation is given 
below. 
\newline
$$
\left[
\begin{matrix}
0 & 1 & 0 & 0 & 1 & 1 & 0 \\
1 & 0 & 0 & 1 & 0 & 0 & 0 \\
1 & 0 & 0 & 0 & 0 & 0 & 1 \\
1 & 0 & 0 & 0 & 0 & 1 & 0 \\
0 & 0 & 0 & 0 & 0 & 1 & 0 \\
0 & 0 & 0 & 0 & 0 & 0 & 0 \\
0 & 0 & 0 & 0 & 0 & 1 & 0 \\
\end{matrix}
\right]
$$

Write down the adjacency lists representation of $G$.

\end{Exercise}

\begin{Exercise}[2003]
\label{ex:listreverse}

Consider the digraph $G$ given by the following adjacency lists
representation.
$$
\begin{tabular}{cccc}
2 &  & \\
0 & &\\
0 & 1&\\   
4 & 5 & 6\\
5 & &\\
3 & 4 & 6 \\
1 & 2 & \\
\end{tabular}
$$

Write down the adjacency lists representation of the reverse digraph
$G_r$.

\end{Exercise}

\begin{Exercise}[2003SC]
\label{ex:divisible}

Consider the digraph $G$ whose nodes are the integers from $1$ to $12$
inclusive and such that $(i, j)$ is an arc if and only if $i$ is a
proper divisor of $j$ (that is, $i$ divides $j$ and $i\neq j$).

Write down the adjacency lists representation of $G$ and of $G_r$.

\begin{Exercise} \label{ex:heaprep} 
Write the adjacency lists
and adjacency matrix representation for a complete binary tree
with $7$ vertices, assuming they are ordered $1, \dots, 7$ as in
Section~\ref{sec:heapsort}.

\end{Exercise}

\end{Exercise}


\section{Implementation of Graph ADT operations}
\label{sec:graphadtimpl}

In this section we discuss the implementation of basic graph operations 
we have seen in Section~\ref{sec:graphdefs}.



A matrix is simply an array of arrays. The lists representaiton is
a list of lists, but a list can be implemented in several ways, for
example an array or singly- or doubly-linked list using pointers. These
have different properties; for example, accessing the middle element is
$\Theta(1)$ for an array but $\Theta(n)$ for a linked list. In any case,
however, to \boldfont{find} a value that may or may not be in the list
requires sequential search and takes $\Theta(n)$ time in the worst case.

We now discuss the comparative performance of some basic oeprations
using the  different data structures.

Suppose we wish to check whether arc $(i,j)$ exists. Using the adjacency
matrix representation, we can perform this operation in constant time,
since we are accessing an array element twice. However, with the lists
representation, we will need to scan along the list associated to $i$
to find $j$. This will take time in $\Theta(d)$, where $d$ is the size
of the list, which equals the outdegree of $i$. In the worst case this
might be $\Theta(e)$ since all nodes but $i$ might be sinks. On the
other hand, for a sparse digraph, the average outdegree is $\Theta(1)$,
so arc lookup can be done on average in constant time. Note that if we
want to print out all arcs of a digraph, this will take time $\Theta(e)$
in the lists case and $\Theta(n^2)$ in the matrix case.

To compute the outdegree takes $\Theta(n)$ with matrix (scan along a
row), and the same is true for indegree. For lists, the outdegree can
be found in time $\Theta(1)$ since a list knows its size. However the
indegree is harder; we must search each list for the node in question,
and this takes $\Theta(e)$. If we wish to compute just one indegree,
this might be acceptable, but if all indegrees are required, this will be
inefficient. It is better to compute the reverse digraph once and then
read off the outdegrees, this last step taking time $\Theta(n)$. How
should we compute the reverse digraph? An obviously good way is to scan
each list and print each entry in the correct list: if we encounter $j$
in list $i$, then $i$ is written to list $j$. Insertion of entries can
be done in constant time (note that the lists produced in this way for
the reverse digraph are naturally sorted). One way around all this work
is to use in our definition of adjacency lists representation, instead
of just the out-nieghbours, a list of in-neighbours also. This may be
useful in some contexts but in general requires more space than is needed.

Adding or deleting an arc is $\Theta(1)$ with the adjacency matrix,
but can take $\Theta(e)$ in the lists case. On the other hand, adding a
node is $\Theta(1)$ in the lists case, since we just add it at the end,
but adding a node with the matrix representation requires us to add aan
extra row and column of zeros, taking $\Theta(n)$ time. Deleting a node
(which perhaps necessitates deleting some arcs) is trickier. In the
matrix case, we must delete a row and column, and move some elements;
this takes $\Theta(n)$ time. In the lists case, we must remove a list
(constant time with a doubly linked list, $\Theta(n)$ otherwise) but
also all references to the deleted node in other lists. This requires
scanning each list for the offending entry and deleting it, which takes
$\Theta(e)$.

We conclude by discussing space requirements. The adjacency matrix
representation requires $\Theta(n^2)$ storage: we simply need $n^2$
bits. It appears that an adjacency lists representation requires
$\Theta(n+e)$ storage, since we must store an endpoint of each arc,
and we need to allocate space for each node's list. However this is
not strictly true for large graphs. Each node number requires some
storage; the number $m$ requires on average  $\Theta(\log m)$ bits. If,
for example, we have a digraph on $n$ nodes where every possible arc
occurs, then the total storage required is of order $n^2 \log n$,
worse than with a matrix representation. For small, sparse digraphs,
it is true that lists use less space than a matrix, whereas for small
dense digraphs the space requirements are comparable. For large sparse
digraphs, a matrix can still be more efficient, but this happens rarely.

The remarks above show that it is not immediately clear which
representation to use. We will mostly use adjacency lists, which are
clearly superior for many common tasks (such as graph traversals, covered
in Chapter~\ref{ch:traversal}) and generally better for sparse digraphs.


Any implementation of an abstract data type (e.g. as a Java class) must
include objects and ``methods".  While most people would include methods
for adding nodes, deleting arcs, and so on, it is not clear where to
draw the line. In the Appendix~\ref{app:javagraph}, one way of writing
Java classes to deal with graphs is presented in detail. There are
obviously a lot of different choices one can make.
