\chapter{Mathematical Background}
\label{ch:app:mathtools}

jo-TODO
MCW-TODO check whether we need anything else, add more examples and self-quiz questions

We list here some important mathematical facts and techniques  that will be used often in this course. It is very important to understand them well - ask for help if your background has substantial holes.

\paragraph{Sets}

A \emph{set} is an unordered collection of objects (called \emph{elements}).

Important sets: 
\begin{itemize}
\item $\mathbb{N} = \{0,1,2,\dots \}$, the set of \emph{natural numbers}.
\item $\mathbb{R}$, the set of \emph{real numbers}.
\item $\varnothing = \{\}$, the \emph{empty set} having no elements.
\end{itemize}

Notation: $X = \{2,3,5,7,11\}$ or $X = \{x\in \mathbb{N} : \text{$x$ is prime and $x< 12$}\}$.
Operations: 
\begin{itemize}
\item $A\cap B = \{x: x\in A \text{ and } x\in B\}$
\item $A\cup B = \{x: x\in A \text{ or  } x\in B\}$
\item $|A|$ is the number of elements of $A$.
\end{itemize}


\paragraph{Functions}

A \emph{function} is a mapping $f$ from a set $X$ (the \emph{domain}) to a set $Y$ (the \emph{codomain}) such that every $x\in X$ maps to a unique $y\in Y$.

Important functions from $\mathbb{R}$ to $\mathbb{R}$: 
\begin{itemize}
\item Power functions $f(x) = x$, $f(x) = x^2$, $f(x) = x^3$, etc
\item Exponential functions $f(x) = 2^x$, $f(x) = (1.5)^x$, etc
\item Logarithm (inverse of exponential) has a different domain.
\end{itemize}

More important functions: \emph{ceiling} rounds up to nearest integer, e.g. $\lceil 3.7 \rceil = 4$. \emph{Floor} rounds down, e.g. $\lfloor 3.7 \rfloor= 3 = \lfloor 3 \rfloor$.

\paragraph{Basic properties of important functions}
\begin{itemize}
\item For $a>1$ the exponential $f(x) = a^x$ is \emph{increasing} and positive, and satisfies $a^{x+y} = a^x a^y$ for all $x,y\in \mathbb{R}$.
\item For $a>1$ the logarithm $\log_a$ is increasing and satisfies $\log_a(xy) = \log_a x + \log_a y$ for all $x,y>0$.
\item We write $\ln = \log_e$ and $\lg = \log_2$. Note that $\log_a x = \log_b x \log_a b$.
\item Derivatives: $\frac{d}{dx} e^x = e^x, \frac{d}{dx} \ln x = \frac{1}{x}$.
\end{itemize}


\paragraph{Sums}

A \emph{sequence} is a function $f:\mathbb{N} \to \mathbb{R}$. 

Notation: $f_0, f_1, f_2, \cdots$ where $f_i = f(i)$.

Sum notation: $f_m+f_{m+1}+\cdots +f_n = \sum_{i=m}^n f_i$.

Some important sums: 
\begin{align*}
\sum_{i=1}^n i & = \frac{n(n+1)}{2} \\
\sum_{i=m}^n a^n&  = \frac{a^{n+1}-a^m}{a-1}.
\end{align*}




\paragraph{Proof by induction}

\begin{Example}
Pove that if $a\neq 1$ then for all  $n\geq 0$
$$
\sum_{i=0}^n a^n = \frac{a^{n+1} - 1}{a - 1}.
$$
Proof: It is true for $n=0$: $1 = 1$. Assume it is true for a given  $n\geq 0$. Then
\begin{align*}
\sum_{i=0}^{n+1} a^i & = a^{n+1} + \sum_{i=0}^n a^i 
= a^{n+1} + \frac{a^{n+1} - 1}{a-1} \\
& = \frac{a^{n+1} (a - 1) + a^{n+1} - 1}{a-1}\\
& = \frac{a^{n+2} - 1}{a-1}.
\end{align*}
Thus the result holds for \emph{all} $n\geq 0$, by \emph{mathematical induction}.
\end{Example}



\paragraph{Binary trees}

A \emph{binary tree} is either empty or consists of a \emph{root} node connected to an ordered pair of binary trees. 
 
We can also define them using explicit external nodes (null pointers) to represent empty trees. Then a binary tree is an external node or an internal node connected to an ordered pair of binary trees.
 
Binary trees can be traversed in several ways,including \emph{preorder},\emph{postorder}, \emph{inorder}. The latter recursively visits the left child, node, and right child.

\paragraph{Limits}

Write $\lim_{x\to \infty} f(x) = \infty$ if for \emph{all} $N>0$ there is \emph{some} point past which $f(x) > N$ for \emph{all} $x$, and write $\lim_{x\to\infty} f(x) = 0$ if for \emph{all} $\varepsilon>0$ there is \emph{some} point past which $f(x) < \varepsilon$ for \emph{all} $x$.

If $0 < L < \infty$ we say $\lim_{x\to\infty} f(x) = L$ if for all $\varepsilon > 0$ there is \emph{some} point beyond which $L - \varepsilon < f(x) < L + \varepsilon$ for \emph{all} $x$.

\begin{Example}
$\lim_{x\to\infty} x^2 = \infty$, $\lim_{x\to \infty} 1/x = 0$.
\end{Example}

Important techqniue: \emph{L'H\^{o}pital's rule}: $\lim_{x\to \infty} f(x)/g(x) = \lim_{x\to \infty} f'(x)/g'(x)$ as long as the latter exists. 

\begin{Example}
 $\lim_{x\to\infty} \frac{e^x}{x} = \lim_{x\to\infty} e^x = \infty$; $\lim_{x\to\infty} (\ln x)/x = \lim_{x\to\infty} (1/x)/1 = 0$.
 \end{Example}



