\setcounter{chapter}{-1}
\chapter{Basic mathematical background}
\label{ch:app:mathtools}

We list here some important mathematical facts and techniques  that will be used often in this course. 
It is very important to understand them well - ask for help if your background has substantial holes.


\section{Sets}
A \defnfont{set} is an unordered collection of distinct objects (repeated elements are not allowed), called \defnfont{elements} of the set. 
We write $a \in X$ to mean that $a$ is an element of $X$. Some important sets are 
\begin{itemize}
\item $\mathbb{N} = \{0, 1, 2, \dots\}$, the set of \defnfont{natural numbers},
\item $\mathbb{R}$, the set of \defnfont{real numbers}, and
\item $\varnothing = \{\}$, the \defnfont{empty set} having no elements.
\end{itemize}

Notation: List elements, e.g. $X = \{2,3,5,7,11\}$ or use set-builder notation $X = \{x \in \mathbb{N} : x \text{ is prime and } x < 12\}$. 
Note that $\{a, a, b\} = \{a, b\}$ because elements cannot be repeated in a set.

Operations on sets: 
\begin{itemize}
\item $A\cap B = \{x: x \in A \text{ and } x \in B\}$, the \defnfont{intersection} of $A$ and $B$.
\item $A\cup B = \{x: x \in A \text{ or  } x \in B\}$, the \defnfont{union} of $A$ and $B$.
\item $|A|$ is the number of elements of $A$, the \defnfont{cardinality} of $A$.
\end{itemize}

\begin{Boxample}[4]
What is the cardinality of $\{1,1,4\}$? Of $\{x^2 : x \in \{-1,1,2\}\}$?
\end{Boxample}


\section{Functions}
A \defnfont{function} is a mapping $f$ from a set $X$ (the \defnfont{domain}) to a set $Y$ (the \defnfont{codomain}) such that every $x \in X$ maps to a unique $y \in Y$. 
We write $f \colon X \to Y$. A function $f \colon X \to Y$ is \defnfont{1--1} if whenever $x \neq x'$, we have $f(x) \neq f(x')$.
 It is \defnfont{onto} if every element of $Y$ has the form $f(x)$ for some $x\in X$. 
 A function that is both 1--1 and onto has an \defnfont{inverse}: $f(x) = y$ if and only if $g(y) = x$. 

Important functions from $\mathbb{R}$ to $\mathbb{R}$: 
\begin{itemize}
\item Power functions $f(x) = x$, $f(x) = x^2$, $f(x) = x^3$, etc.
\item Exponential functions $f(x) = 2^x$, $f(x) = (1.5)^x$, etc.
\item Logarithm (inverse of exponential) is defined by $\log_a (y) = x$ if and only if $y = a^x$. 
Its domain is $\{x\in \mathbb{R} : x > 0\}$. We usually just write $\log_a y$, omitting parentheses, if there is no confusion about what is meant.
\end{itemize}

Other useful functions: 
\begin{itemize}
\item \defnfont{Ceiling} rounds up to nearest integer, e.g. $\lceil 3.7 \rceil = 4 = \lceil 4 \rceil$.
\item \defnfont{Floor} rounds down, e.g. $\lfloor 3.7 \rfloor = 3 = \lfloor 3 \rfloor$.
\end{itemize}


\section{Basic properties of important functions}
\begin{itemize}
\item For fixed $a>1$ the exponential $f(x) = a^x$ is \boldfont{strictly increasing} and positive, 
and satisfies $$a^{x+y} = a^x a^y$$ for all $x,y\in \mathbb{R}$. 
\item For fixed $a>1$ the logarithm $\log_a$ is strictly increasing and satisfies 
$$\log_a(xy) = \log_a x + \log_a y$$ for all $x,y>0$.
\item We write $\ln = \log_e$ and $\lg = \log_2$. Note that $\log_a x = \log_a b \log_b x$ and $a^x = e^{x\ln a}$.
\item Derivatives: $(e^x)' := \frac{d}{dx} e^x = e^x, \frac{d}{dx} \ln x = \frac{1}{x},  \frac{d}{dx} x^a = ax^{a-1}$ if $a$ is a constant.
\end{itemize}

\begin{Boxample}[4]
Prove that for every $n\in \mathbb{N}$ such that $n\geq 1$,
$$ \lceil \lg (n+1) \rceil = 1 + \lfloor \lg n \rfloor\text{.}$$
\end{Boxample}


\section{Sums}
A \defnfont{sequence} is a function $f \colon \mathbb{N} \to \mathbb{R}$. 

Notation: $f_0, f_1, f_2, \cdots$ where $f_i = f(i)$.

Sum notation: $f_m+f_{m+1}+\cdots +f_n = \sum_{i=m}^n f_i$.

Some important sums: 
\begin{align*}
	\sum_{i=1}^n i & = \frac{n(n + 1)}{2} \\
	\sum_{i=m}^n a^n&  = \frac{a^{n+1} - a^m}{a - 1}.
\end{align*}

\begin{Boxample}[4]
Simplify $$\sum_{i=1}^{2n} 2^i\text{.}$$
\end{Boxample}


\section{Proof by induction}
\defnfont{Mathematical induction} is a standard technique for proving a collection of results, one for each natural number. 
If we can prove for $n = 0$, and also show that whenever the result we want is true for $n$, it is true for $n+1$,
then we know it must be true for all $n \in \mathbb{N}$.

\begin{Boxample}
Prove that if $a\neq 1$ then for all  $n\geq 0$
$$
	\sum_{i=0}^n a^n = \frac{a^{n+1} - 1}{a - 1}.
$$
Proof: It is true for $n=0$: $1 = 1$. Assume it is true for a given  $n\geq 0$. Then
\begin{align*}
\sum_{i=0}^{n+1} a^i 
	& = a^{n+1} + \sum_{i=0}^n a^i = a^{n+1} + \frac{a^{n+1} - 1}{a - 1} \\
	& = \frac{a^{n+1} (a - 1) + a^{n+1} - 1}{a - 1}\\
	& = \frac{a^{n+2} - 1}{a - 1}.
\end{align*}
Thus the result holds for \boldfont{all} $n \geq 0$, by \boldfont{mathematical induction}.
\end{Boxample}

\begin{Boxample}[4]
Consider the sequence defined by $a_0 = 1$ and $a_n = 3a_{n-1} - 4$ for $n > 0$. Prove that $a_n$ is odd for all $n \in \mathbb{N}$.
\end{Boxample}


\section{Binary trees}
A \defnfont{binary tree} is an object that is either empty or consists of a \defnfont{root} node connected to an ordered pair of binary trees. 
 
We can also define them using explicit external nodes (null pointers) to represent empty trees. Then a binary tree is an external node or an internal node connected to an ordered pair of binary trees.

There is a unique path from the root to each node. The length of this path is the \defnfont{depth} of the node. The maximum depth of all nodes is the \defnfont{height} of the tree.

Binary trees can be traversed in several ways,including \defnfont{preorder}, \defnfont{postorder}, \defnfont{inorder}. 
The latter recursively visits the left child, node, and right child.


\section{Limits}
Write $\lim_{x \to \infty} f(x) = \infty$ if for \boldfont{all} $N>0$ there is \boldfont{some} point past which $f(x) > N$ for \boldfont{all} $x$, 
and write $\lim_{x\to\infty} f(x) = 0$ if for \boldfont{all} $\varepsilon>0$ there is \boldfont{some} point past which $f(x) < \varepsilon$ for \boldfont{all} $x$.

If $0 < L < \infty$ we say $\lim_{x \to \infty} f(x) = L$ if for all $\varepsilon > 0$ there is \boldfont{some} point 
beyond which $L - \varepsilon < f(x) < L + \varepsilon$ for \boldfont{all} $x$.

\begin{Boxample}
$\lim_{x \to \infty} x^2 = \infty$, $\lim_{x \to \infty} 1/x = 0$, $\lim_{x\to \infty} \frac{x-1}{x+1} = 1$.
\end{Boxample}

An important technique is the \defnfont{L'H\^{o}pital's rule}: $$\lim_{x\to \infty} f(x)/g(x) = \lim_{x\to \infty} f'(x)/g'(x)$$ as long as the latter exists. 

\begin{Boxample}
$\lim_{x\to\infty} \frac{e^x}{x^2} =  \lim_{x\to\infty} \frac{e^x}{2x} = \lim_{x\to\infty} \frac{e^x}{2} = \infty$.
\end{Boxample}
 
\begin{Boxample}[5]
What is  $\lim_{x\to\infty} \frac{\ln x}{x}$?
\end{Boxample}



