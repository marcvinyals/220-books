\setcounter{chapter}{-1}
\chapter{Basic Mathematical Background}
\label{ch:app:mathtools}

MCW-TODO check whether we need anything else, add more examples and self-quiz questions using stuff from CS120

We list here some important mathematical facts and techniques  that will be used often in this course. It is very important to understand them well - ask for help if your background has substantial holes.

\section{Sets}

A \defnfont{set} is an unordered collection of distinct objects (called \defnfont{elements}). We write $a\in X$ to mean that $a$ is an element of $X$.

Important sets: 
\begin{itemize}
\item $\mathbb{N} = \{0,1,2,\dots \}$, the set of \defnfont{natural numbers}.
\item $\mathbb{R}$, the set of \defnfont{real numbers}.
\item $\varnothing = \{\}$, the \defnfont{empty set} having no elements.
\end{itemize}

Notation: list elements, e.g. $X = \{2,3,5,7,11\}$ or use set-builder notation $X = \{x\in \mathbb{N} : \text{$x$ is prime and $x< 12$}\}$. Note that $\{a, a, b\} = \{a, b\}$ because elements cannot be repeated in a set.

Operations: 
\begin{itemize}
\item $A\cap B = \{x: x\in A \text{ and } x\in B\}$, the \defnfont{intersection} of $A$ and $B$;
\item $A\cup B = \{x: x\in A \text{ or  } x\in B\}$, the \defnfont{union} of $A$ and $B$;
\item $|A|$ is the number of elements of $A$, the \defnfont{cardinality} of $A$.
\end{itemize}


\section{Functions}

A \defnfont{function} is a mapping $f$ from a set $X$ (the \defnfont{domain}) to a set $Y$ (the \defnfont{codomain}) such that every $x\in X$ maps to a unique $y\in Y$.

Important functions from $\mathbb{R}$ to $\mathbb{R}$: 
\begin{itemize}
\item Power functions $f(x) = x$, $f(x) = x^2$, $f(x) = x^3$, etc
\item Exponential functions $f(x) = 2^x$, $f(x) = (1.5)^x$, etc
\item Logarithm (inverse of exponential) is defined by $\log_a (y) = x$ if and only if $y = a^x$. Its domain is $\{x\in \mathbb{R} : x > 0\}$. We usually just write $\log_a y$, omitting parentheses, if there is no confusion about what is meant.
\end{itemize}

Other useful functions: 
\begin{itemize}
\item \defnfont{Ceiling} rounds up to nearest integer, e.g. $\lceil 3.7 \rceil = 4 = \lceil 4 \rceil$. 
\item \defnfont{Floor} rounds down, e.g. $\lfloor 3.7 \rfloor= 3 = \lfloor 3 \rfloor$.
\end{itemize}



\section{Basic properties of important functions}
\begin{itemize}
\item For fixed $a>1$ the exponential $f(x) = a^x$ is \defnfont{increasing} and positive, and satisfies $$a^{x+y} = a^x a^y$$ for all $x,y\in \mathbb{R}$. 
\item For fixed $a>1$ the logarithm $\log_a$ is increasing and satisfies 
$$\log_a(xy) = \log_a x + \log_a y$$ for all $x,y>0$.
\item We write $\ln = \log_e$ and $\lg = \log_2$. Note that $\log_a x = \log_a b \log_b x$ and $a^x = e^{x\ln a}$.
\item Derivatives: $\frac{d}{dx} e^x = e^x, \frac{d}{dx} \ln x = \frac{1}{x},  \frac{d}{dx} x^a = ax^{a-1}$ if $a$ is a constant.
\end{itemize}

\begin{Boxample}[4]
Prove that for every $n\in \mathbb{N}$ such that $n\geq 1$,
$$
\lceil \lg (n+1) \rceil = 1 + \lfloor \lg n \rfloor.
$$

\end{Boxample}

\section{Sums}

A \defnfont{sequence} is a function $f:\mathbb{N} \to \mathbb{R}$. 

Notation: $f_0, f_1, f_2, \cdots$ where $f_i = f(i)$.

Sum notation: $f_m+f_{m+1}+\cdots +f_n = \sum_{i=m}^n f_i$.

Some important sums: 
\begin{align*}
\sum_{i=1}^n i & = \frac{n(n+1)}{2} \\
\sum_{i=m}^n a^n&  = \frac{a^{n+1}-a^m}{a-1}.
\end{align*}

\begin{Boxample}[4]
Simplify 
$$
\sum_{i=1}^{2n} 2^i.
$$
\end{Boxample}

\section{Proof by induction}
A standard technique for proving a collection of results, one for each natural number. If we can prove for $n=0$, and also show that whenever the result we want is true for $n$, it is true for $n+1$, then we know it must be true for all $n\in \mathbb{N}$.

\begin{Example}
Prove that if $a\neq 1$ then for all  $n\geq 0$
$$
\sum_{i=0}^n a^n = \frac{a^{n+1} - 1}{a - 1}.
$$
Proof: It is true for $n=0$: $1 = 1$. Assume it is true for a given  $n\geq 0$. Then
\begin{align*}
\sum_{i=0}^{n+1} a^i & = a^{n+1} + \sum_{i=0}^n a^i 
= a^{n+1} + \frac{a^{n+1} - 1}{a-1} \\
& = \frac{a^{n+1} (a - 1) + a^{n+1} - 1}{a-1}\\
& = \frac{a^{n+2} - 1}{a-1}.
\end{align*}
Thus the result holds for \defnfont{all} $n\geq 0$, by \defnfont{mathematical induction}.
\end{Example}



\section{Binary trees}

A \defnfont{binary tree} is an object that is either empty or consists of a \defnfont{root} node connected to an ordered pair of binary trees. 
 
We can also define them using explicit external nodes (null pointers) to represent empty trees. Then a binary tree is an external node or an internal node connected to an ordered pair of binary trees.
 
Binary trees can be traversed in several ways,including \defnfont{preorder}, \defnfont{postorder}, \defnfont{inorder}. The latter recursively visits the left child, node, and right child.

\section{Limits}

Write $\lim_{x\to \infty} f(x) = \infty$ if for \defnfont{all} $N>0$ there is \defnfont{some} point past which $f(x) > N$ for \defnfont{all} $x$, and write $\lim_{x\to\infty} f(x) = 0$ if for \defnfont{all} $\varepsilon>0$ there is \defnfont{some} point past which $f(x) < \varepsilon$ for \defnfont{all} $x$.

If $0 < L < \infty$ we say $\lim_{x\to\infty} f(x) = L$ if for all $\varepsilon > 0$ there is \defnfont{some} point beyond which $L - \varepsilon < f(x) < L + \varepsilon$ for \defnfont{all} $x$.

\begin{Example}
$\lim_{x\to\infty} x^2 = \infty$, $\lim_{x\to \infty} 1/x = 0$.
\end{Example}

Important technique: \defnfont{L'H\^{o}pital's rule}: $\lim_{x\to \infty} f(x)/g(x) = \lim_{x\to \infty} f'(x)/g'(x)$ as long as the latter exists. 

\begin{Example}
 $\lim_{x\to\infty} \frac{e^x}{x} = \lim_{x\to\infty} e^x = \infty$; $\lim_{x\to\infty} (\ln x)/x = \lim_{x\to\infty} (1/x)/1 = 0$.
 \end{Example}
 
 \begin{Boxample}[5]
What is  $\lim_{x\to\infty} \frac{\log x}{x}$?
 
 \end{Boxample}



